\begin{problem}
  For each of the following languages, classify the language
  into one of the following categories:
  (a) unrecognizable;
  (b) recognizable, but not decidable;
  (c) decidable.
  \begin{enumalph}
    \item $L_1 = \set{\vector{M} :
      \text{$M$ is a TM and $M$ accepts at least two strings.}}$
      \begin{Answer}
        \begin{claim}
          $L$ is recognizable but not decidable.

          \begin{proof}
            Since $\Sigma^*$ is decidable, it is also lexicographically enumerable.
            Let $E$ be an lexicographic enumerator TM for $\Sigma^*$.\\
            Construct a new TM $M'$ as follows:

            \step
            $M'$ = ``On input $\vector{M}$:
              \begin{enumarabic}
                \item Define $A = \emptyset$.
                \item For $i \gets 1, 2, 3, \ldots$
                  \begin{enumarabic*}
                    \item Define $s_i$ to be the next string in the enumeration of $\Sigma^*$.
                    \item Define $S = \set{s \in \set{s_1, s_2, \ldots s_i} : s \notin A}$.
                    \item Simulate $M$ for $i$ steps on each $s \in S$.
                    \item If $M$ accepts some $s_k \in S$
                    \begin{enumarabic*}
                      \item $A \gets A \cup \set{s_k}$
                      \item If $\abs{A} \geq 2$, \Accept.''
                    \end{enumarabic*}
                  \end{enumarabic*}
              \end{enumarabic}

            \step
            Given a TM $M$ in $L$, then $M'$ will always eventually halt
            after $M$ accepts at least two strings.
            However, given a TM not in $L$ -- that is, a TM $M$ that accepts
            less than $2$ strings -- $M'$ will never halt.
            There is no work-around for this problem, since at any given
            instant there is yet another string in $\Sigma^*$ that has not yet been
            tested by $M'$ to determine whether $M$ accepts it.
          \end{proof}
        \end{claim}
      \end{Answer}

    \newpage
    \item $L = \set{\vector{M} :
      \text{$M$ is a TM and $M$ accepts exactly two strings.}}$
      \begin{Answer}
        Note that this is a subset of the set in part (a).
        \begin{claim}
          $L$ is unrecognizable.
  
          \begin{proof}
            Suppose $L$ were recognizable, then there exists a TM $M'$ that,
            given a turing machine $M$, returns \Accept if $M$ accepts
            exactly two strings.

            \step
            For instance, take $M_1$, a TM defined over $\Sigma = \set{0, 1}$
            as follows:

            \step
            $M_1$ = ``On input $w$:
              \begin{enumarabic}
                \item If $w = \eps$ or $w = 11$, \Accept.
                \item Otherwise, \Reject.''
              \end{enumarabic}

            \step
            Clearly, $M_1$ is a decider that accepts exactly two strings:
            $\eps$ and $11$. So $M'$ should accept $M_1$.
            However, given only the definition of $M_1$,
            even when $M'$ has determined that $M_1$ accepts
            $\eps$ and $11$, $M'$ has not yet determined if
            $M_1$ rejects all remaining strings in $\Sigma^*$
            --- and to determine that it needs to enumerate all the members of
            $\Sigma^*$ and confirm that each is rejected by $M_1$.
            $\Sigma^*$ is an infinite set, so $M'$ will always have another 
            string to test, and will never halt and return \Accept.

            \step
            So, ironically, $M'$, a TM recognizing TMs that accept exactly
            two strings, never determines enough information to conclusively
            accept a TM that does accept exactly two strings.
          \end{proof}
        \end{claim}
      \end{Answer}

    \newpage
    \def \ALLtm { \textsf{ALL}_{\textsf{TM}} }
    \def \Etm { \textsf{E}_{\textsf{TM}} }
    
    \item $L = \set{\vector{M_1, M_2} :
      \text{$M_1$ and $M_2$ are TMs over the input alphabet $\set{0, 1}$
      and $\cL(M_1) = \set{0, 1}^* - \cL(M_2)$.}}$
      \begin{Answer}
        \begin{claim}
          $L$ is not recognizable.

          \begin{proof}
            Suppose $L$ were recognizable; that is, there exists some TM
            $M$ that, given two TMs $M_1$ and $M_2$, returns \Accept
            if $\cL(M_1) = \set{0, 1}^* - \cL(M_2)$.

            \step
            We show that this cannot happen.
            First, define a TM $M_2$ recognizing
            the set
            \[ \ALLtm = 
                \set{\vector{M} : \text{$M$ is a TM and $M$ accepts all strings in $\set{0, 1}^*$}}. \]

            \step
            $M_2$ = ``On input $w$:
              \begin{enumarabic}
                \item If input is in $\set{0, 1}^*$, \Accept, otherwise \Reject.''
              \end{enumarabic}

            \step
            Clearly, $\cL(M_2) = \set{0, 1}^*$. Consider the problem of deciding
            $\Etm = \set{\vector{M} : \text{$M$ is a TM and $\cL(M) = \emptyset$}}$.
            $\Etm$ is the complement of $\ALLtm$ over the alphabet $\set{0, 1}$.
            Therefore, we can recognize $\Etm$ as follows:

            \step
            $M_3$ = ``On input $\vector{M}$, where $M$ is a TM:
              \begin{enumarabic}
                \item Simulate $M_1$ on input $\vector{M, M_2}$.
                \item If $M_1$ accepts, \Accept.
                \item If $M_1$ rejects, \Reject.''
              \end{enumarabic}

            \step
            $M_3$ therefore decides $\Etm$.
            But we know $\Etm$ is undecidable, so the initial assumption that
            $M_1$ recognizes $L$ must be wrong to begin with, and $L$ must
            not be recognizable.
          \end{proof}
        \end{claim}
      \end{Answer}
  \end{enumalph}
\end{problem}
