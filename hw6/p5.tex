\def \Udfa { \textsf{U}_{\textsf{DFA}} }
\def \Upda { \textsf{U}_{\textsf{PDA}} }
\def \Utm  { \textsf{U}_{\textsf{TM}}  }
\def \Etm  { \textsf{E}_{\textsf{TM}}  }

\begin{problem}
  All of the automata models we have studied in this course allow 
  ``useless'' states, i.e., states which are never entered in any run.
  It would be nice to have an algorithm that could detect and prune
  such useless states in an automaton, but this is not always possible!
  Define the languages
  \begin{align*}
    \Udfa = \set{\vector{M} :
      \text{$M$ is a DFA and $M$ has at least one useless state}}, \\
    \Upda = \set{\vector{M} :
      \text{$M$ is a PDA and $M$ has at least one useless state}}, \\
    \Utm = \set{\vector{M} :
      \text{$M$ is a TM and $M$ has at least one useless state}}.
  \end{align*}
  
  \step
  Prove that $\Udfa$ and $\Upda$ are decidable.
\end{problem}
\begin{Answer}
  Given a DFA or PDA $M$, we can determine if $M$ does not accept any string
  in $\Sigma$ by doing a simple Graph search from the starting state
  $q_0$. If $\cL(M) = \emptyset$ then there is no computational path
  (sequence of transitions) starting from the start state, that
  leads to an accepting state.

  % Define a TM $E_{\textsf{DFA}}$ that decides the set
  % \[ \set{\vector{M} : \text{$M$ is a DFA or a PDA and  $\cL(M) = \emptyset$}} \]
  % as follows:

  \step
  \begin{multicols*}{2}

    \step
    $E_{\textsf{DFA}}$ = ``On input $\vector{M}$:
      \begin{enumarabic}
        \item If $q_0 \in F$, \Reject.
        \item $queue \gets [q_0]$.
        \item $V \gets \emptyset$.
        \item While $\abs{queue} \neq 0$
          \begin{enumarabic*}
            \item $q \gets$ first item removed from $queue$.
            \item $V \gets V \cup \set{q}$.
            \item For each $a \in \Sigma$
              \begin{enumarabic*}
                \item $q' \gets \delta(q, a)$.
                \item If $q' \in F$, \Reject.
                \item If $q' \notin V$; Add $q'$ to $queue$.
              \end{enumarabic*}
          \end{enumarabic*}
        \item If no accepting state reached, \Accept.''
      \end{enumarabic}
      
  
      \step
      $E_{\textsf{PDA}}$ = ``On input $\vector{M}$:
        \begin{enumarabic}
          \item If $q_0 \in F$, \Reject.
          \item $queue \gets [q_0]$.
          \item $V \gets \emptyset$.
          \item While $\abs{queue} > 0$
            \begin{enumarabic*}
              \item $q \gets queue[0]$.
              \item $queue \gets queue[1:]$.
              \item $V \gets V \cup \set{q}$.
              \item $Q' \gets \set{\delta(q, \sigma, \gamma) : \sigma \in \Sigma, \gamma \in \Gamma}$.
              \item If $Q' \cap F \neq \emptyset$, \Reject.
              \item $queue \gets queue \cup \set{q \in Q' : q \notin V}$.
            \end{enumarabic*}
          \item If no accepting state reached, \Accept.''
        \end{enumarabic}
  \end{multicols*}
  
  Since any given DFA or PDA has a finite number of states,
  a finite language alphabet, and, in the case of a PDA, a finite
  stack alphabet, both $E_{\textsf{DFA}}$ and $E_{\textsf{PDA}}$
  eventually halt with either \Accept or \Reject.

  \newpage
  We can define a TM that decides $\Udfa$ and $\Upda$ as follows:

  \step
  $M'$ = ``On input $\vector{M}$ where M is a PDA or a DFA:
    \begin{enumarabic}
      \item Mark all the accepting states of $M$ as non-accepting.
      \item For each state $q \in Q$ where $Q$ is the set of states
        in $M$ in $M$:
        \begin{enumarabic*}
          \item Mark $q$ as accepting.
          \item If $M$ is a DFA, run $E_{\textsf{DFA}}$ on $\vector{M}$.
            If $M$ is a PDA, run $E_{\textsf{PDA}}$ on $\vector{M}$.
          \item If $E_{\textsf{DFA}}$ or $E_{\textsf{PDA}}$ accepts $M$,
            then $q$ is a useless state. \Accept.
          \item If $E_{\textsf{DFA}}$ or $E_{\textsf{PDA}}$ rejects $M$,
            then $q$ is not a useless state.
            Un-mark $q$ as accepting and repeat for the next state.
        \end{enumarabic*}
      \item If no state is found to be useless, \Reject.''
    \end{enumarabic}
\end{Answer}
