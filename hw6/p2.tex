\def \half {\textsc{Half }}
\begin{problem}
  Furthermore, recognizable languages are closed under (a) union,
  (b) intersection, (c) concatenation, (d) Kleene star, and
  (e) the \half operation. Prove the following:
  \begin{enumalph}
    \item union
    \begin{Answer}
      Let $T_1$ and $T_2$ be TMs recognizing $L_1$ and $L_2$ respectively.
      Construct a $2$-tape TM $T$ as follows:

      \step
      $T$ = ``On input $w$:
        \begin{enumarabic}
          \item Copy $w$ onto the second tape.
          \item Simulate $T_1$ on input $w$ on tape $1$
            and concurrently simulate $T_2$ on input $w$ on the second tape.
          \item If $T_1$ halts and accepts, \Accept.
          \item If $T_2$ halts and accepts, \Accept.
          \item Otherwise, keep running $T_1$ and $T_2$.''
        \end{enumarabic}

        \step
        If $w \in L_1 \cup L_2$, then either $T_1$ will eventually halt
        and accept or $T_2$ will eventually halt and accept,
        so $T$ will eventually halt and accept.
        However, if $w \notin L_1 \cup L_2$, then $T_1$ and $t_2$ may
        never halt, so $T$ may also never halt.
        Therefore, $T$ recognizes, but does not decide, $L_1 \cup L_2$.  \qed
    \end{Answer}

    \newpage
    \setcounter{enumi}{2}
    \item concatenation
    \begin{Answer}
      Let $T_1$ and $T_2$ be TMs recognizing $L_1$ and $L_2$ respectively.
      Construct an NDTM $T$ as follows:

      \step
      $T$ = ``On input $w$:
        \begin{enumarabic}
          \item Insert a new symbol $\# \notin \Sigma$ before the first
            symbol of $w$.
          \item Non-deterministically do one of the following:~\label{step:2.1.2}
            \begin{enumarabic*}
              \item Simulate $T_1$ on the contents of $w$ before the $\#$
                symbol and simulate $T_2$ on the contents of $w$ after
                the $\#$ symbol. If both accept, \Accept.
              \item Shift the $\#$ symbol to the right by $1$ and restart
                step~\ref{step:2.1.2}''
            \end{enumarabic*}
        \end{enumarabic}

        \step
        If $T$ accepts a string $w$, then there must exist some prefix of $w$
        accepted by $T_1$ such that the remainder of $w$ is accepted by $T_2$.
        Therefore, $w \in L_1 L_2$.
        Similarly, if $w \in L_1 L_2$, then it can be split into two strings
        $w_1$ and $w_2$ such that $w_1 \in L_1$ and $w_2 \in L_2$,
        so $T$ will accept $w$.

        However, since $T_1$ and $T_2$ may not halt on inputs not in $L_1$
        and $L_2$ respectively, $T$ may not halt on inputs not in $L_1 L_2$,
        so it does not decide $L_1 L_2$.  \qed
    \end{Answer}

    \newpage
    \setcounter{enumi}{4}
    \item \half
    \begin{Answer}
      Let $L$ be a language over an alphabet $\Sigma$,
      and let $T$ be a TM recognizing $L$.

      \step
      First, construct an enumerator TM
      $N$ that enumerates members of $\Sigma$ of a specific length.

      \step
      $N$ = ``On input $n$:
        \begin{enumarabic}
          \item $l \gets 0$.
          \item while $l <= n$:
            \begin{enumarabic*}
              \item Determine the next member $x \in \Sigma^*$ (in lexicographic ordering).
              \item If $\abs{x} = n$, \List $x$.
              \item $l \gets \abs{x}$.''
            \end{enumarabic*}
        \end{enumarabic}

      \step
      To recognize $\half(L)$, we construct a TM $T'$ as follows:

      \step
      $T'$ = ``On input $w$:
        \begin{enumarabic}
          \item $n \gets \abs{w}$.
          \item Simulate $N$ on input $n$.
          \item For each $x \in \Sigma^*$ enumerated by $N$:
            \begin{enumarabic*}
              \item Simulate $T$ on $wx$.
              \item If $T$ accepts, \Accept.
            \end{enumarabic*}
        \end{enumarabic}
    \end{Answer}
  \end{enumalph}
\end{problem}
