\begin{problem}
  Give a simple CFG that generates the language
  $L_3 = \set{x \in \set{0, 1}^* : x \ne x^R}$.

  \step
  Formally prove that your CFG is sound and complete.
\end{problem}
\begin{Answer}
  \begin{align*}
    \underline{G_3:} \\
    S &\derives 0S0 \mid 0B1 \mid 1B0 \mid 1S1 \\
    B &\derives 0B0 \mid 0B1 \mid 1B0 \mid 1B1 \mid 0 \mid 1 \mid \eps
  \end{align*}

  \step
  \begin{claim}
    $\calL(G_3) = L_3$.

    \begin{proof}
      We need to show that $G_3$ is both complete $\calL(G_3) \supseteq L_3$
      and $\calL(G_3) \subseteq L_3$.

      \step
      \textbf{Completeness:} \\
      Let $x \in L_3$, and write $x = x_1, x_2, \ldots, x_n$
      with each $x_i \in \set{0, 1}$.
      
      \step
      By definition of $L_3$, if $x \in L_3$ then $x \ne x^R$.
      Therefore, $x_1x_2 \ldots x_n \ne x_nx_{n-1} \ldots x_1$,
      implying that $x_i \ne x_{n-i+1}$ for some $i \in \set{1, 2, \ldots, \floor{n/2}}$.
      Therefore, the sequence $x_i, x_i+1, \ldots, x_{n-i}$
      can be written as either \crim{$0y1$} or \crim{$1y0$}, where $y \in \set{0, 1}^*$.

      \step
      Back to our CFG, note that the rule
      \crim{`$B \derives 0B0 \mid 0B1 \mid 1B0 \mid 1B1 \mid 0 \mid 1 \mid \eps$'}
        can derive \emph{any} string in $\set{0, 1}^*$.
        Therefore, if we can a string in $L_3$ as either $0B1$ or $1B0$
        then we can yield it from $B$.

      \step
      The starting symbol of $G_3$ is $S$,
      and the rules for $S$ are \crim{$S \derives 0S0 \mid 0B1 \mid 1B0 \mid 1S1$}.
      $S$ either yields:
      \begin{itemize}
        \item $0S0$ or $1S1$ if the two symbols at the ends of the string are the same.
          However, we may never yield a full string from only yielding a sequence of $S$'s,
          since $S$ does not have a rule that removes all symbols not in $\set{0, 1}^*$
          from the string.
        \item $0B1$ or $1B0$ \emph{only if} the two symbols at the ends of the string
          are different. By yielding $B$, we can eventually 
      \end{itemize}
    \end{proof}
  \end{claim}
\end{Answer}
