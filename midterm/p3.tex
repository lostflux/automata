\begin{problem}
  Give a simple CFG that generates the language
  $L_3 = \set{x \in \set{0, 1}^* : x \ne x^R}$.

  \step
  Formally prove that your CFG is sound and complete.
\end{problem}
\begin{Answer}
  \begin{align*}
    \underline{G_3:} \\
    S &\derives 0S0 \mid 0B1 \mid 1B0 \mid 1S1 \\
    B &\derives 0B0 \mid 0B1 \mid 1B0 \mid 1B1 \mid 0 \mid 1 \mid \eps
  \end{align*}

  \step
  \begin{claim}
    $\calL(G_3) = L_3$.

    \begin{proof}
      We need to show that $G_3$ is both complete $\calL(G_3) \supseteq L_3$
      and $\calL(G_3) \subseteq L_3$.

      \step
      \textbf{Observations: } \\
      Let $x = x_1, x_2, \ldots, x_n$ with each $x_i \in \set{0, 1}$.
      By definition of $L_3$, if $x \in L_3$ then $x \ne x^R$.
      Therefore, $x_1x_2 \ldots x_n \ne x_nx_{n-1} \ldots x_1$,
      implying that $x_i \ne x_{n-i+1}$ for some $i \in \set{1, 2, \ldots, \floor{n/2}}$.

      \step
      \textbf{Completeness:} \\
      Let $x \in L_3$, and write $x = x_1, x_2, \ldots, x_n$
      with each $x_i \in \set{0, 1}$.

      \step
      The starting symbol in $G_3$ is $S$.
      Let $i$ be the smallest index such that $x_i \ne x_{n-i+1}$.
      Here's how we can derive $x$ from $S$:
      \begin{itemize}
        \item For each $k \le i$, we have that $x_k = x_{n-k}$,
          so we can write $x_k x_{k+1} \ldots x_{n-k}$ as either $0S0$ or $1S1$.
        \item At $i$, we cannot write $x_i x_{i+1} \ldots x_{n-i}$
          as $0S0$ or $1S1$, since $x_i \ne x_{n-i+1}$.
          Instead, write $x_i x_{i+1} \ldots x_{n-i}$ as either $0B1$ or $1B0$.
        \item For each $i < l \le \floor{n/2}$, we do not really care whether $x_l = x_{n-l}$
          since we already found an index $i$ such that $x_i \ne x_{n-i+1}$,
          which guarantees that $x \ne x^R$.
          Follow the derivations for $B$ to write $x_l x_{l+1} \ldots x_{n-l}$
          as one of $0B1, 1B0, 0B0, 1B1$.
        \item Finally, at index $m = \floor{n/2} + 1$:
          \begin{itemize}
            \item If $m = n - m + 1$, then we have an odd-length string
              and we are currently at the middle element. Yield either $0$ or $1$
              depending on the symbol in the middle of $x_m$.
            \item Otherwise, the string has even length
              and we have already generated strings of length $n/2$
              to the left and to the right of the current position,
              so we yield the empty string.
          \end{itemize}
      \end{itemize}
      Since every string $s \in L_3$ has \emph{at least} one index $i$
      such that $x_i \ne x_{n-i+1}$, we can always make the derivation
      $S \derives 0B1$ or $S \derives 1B0$ at some point in the derivation of $s$,
      and thereafter yield the full string containing only symbols from $\set{0, 1}$
      since since $B$ can generate any string.
      Therefore, any string in $L_3$ can be generated by $G_3$.

      \step
      \textbf{Soundness:} \\
      Let $x = x_1, x_2, \ldots, x_n$ be a string accepted by $G_3$.
      The starting symbol in $G_3$ is $S$,
      while only $B$ can generate strings containing symbols exclusively from $\set{0, 1}$.
      The only valid rules that yield $B$ from $S$ are
      $S \derives 0B1$ and $S \derives 1B0$.
      But $x$ must have at least one index $i$ such that $x_i \ne x_{n-i+1}$
      for the derivation $S \derives 0B1$ or $S \derives 1B0$ to be valid.
      This implies that $x$ must not be equal to $x^R$,
      since the symbol at index $i$ in $x$ is $x_i$ and the symbol at index
      $i$ in $x^R$ is $x_{n-i+1}$ and the two are not equal. 
      Therefore, if $x$ is accepted by $G_3$, then
      it must be the case that $x \ne x^R$ so $x \in L_3$.
    \end{proof}
  \end{claim}
\end{Answer}
