\begin{problem}
  Every language falls into into one of the following three categories:
  \begin{enumroman}
    \item regular
    \item context-free but not regular
    \item not context-free
  \end{enumroman}
  Which of these categories is the following language in?

  \[
    L_6 = \set{ x \in \set{0, 1}^* : \exists y, z \in \set{0, 1}^*
    \text{ such that } x = yz, \abs{y} = \abs{z},
    \text{ and $\beta(y) \equiv 0 \pmod 3 $}}.
  \]
  \emph{Reminder:} For a string $x \in \set{0, 1}^*$, $\beta(x)$ is the
  string $x$ interpreted as a binary number.
  For instance, $\beta(11001) = 25$ and $\beta(0011) = 3$.
  We also define $\beta(\eps) = 0$.
\end{problem}
\begin{Answer}
  $L_6$ is context-free \emph{but not regular}.\\
  It is not regular because the requirement to track length of the string
  and check divisibility by $3$ of the first half of the string
  makes it impossible to reuse states in any DFA recognizing $L_6$.
  Every string is in its own equivalence class,
  which is uniquely determined not just by the length of the string
  but also by the contents of the string.
  Since $L_6$ is an infinite language, any DFA
  would need infinite states to recognize $L_6$,
  yet a DFA must have a finite number of states.

  \step
  However, with the use of a stack, we can easily track the length of
  strings by pushing to the stack while checking divisibility by $3$,
  and afterward pop from the stack while processing the second part of
  the string.
  
  \step
  We can construct a PDA to recognize $L_6$ as follows:
  \begin{figure}[H]
    \centering
    \begin{tikzpicture}
      \green{
        \node[state, initial] (qs) at (0, 0) {$s$};
      }
      \crim{
        \node[state, right of=qs] (q0) at (0, 0) {$q_0$};
        \node[state, right of=q0] (q1) {$q_1$};
        \node[state, right of=q1] (q2) {$q_2$};
        \draw (q0) edge[loop above, above] node {$0, \eps \to \#$} (q0)
              (q0) edge[bend left, above] node {$1, \eps \to \#$} (q1)
              (q1) edge[bend left, above] node {$0, \eps \to \#$} (q2)
              (q1) edge[bend left, below] node {$1, \eps \to \#$} (q0)
              (q2) edge[bend left, below] node {$0, \eps \to \#$} (q1)
              (q2) edge[loop right, right] node {$1, \eps \to \#$} (q2);
      }
      \zaff{
        \node[state, below of=q0] (ql) {$q_{loop}$};
        \draw (ql) edge[loop left, left] node
          {$\begin{aligned} 0, \# \to \eps \\ 1, \# \to \eps \end{aligned}$} (ql);
      }
      \green{
        \node[state, right of=ql, accepting] (qf) {$q_{accept}$};
        \draw (q0) edge[left] node {$\eps, \eps \to \eps$} (ql)
              (qs) edge[above] node {$\eps, \eps \to \$$} (q0)
              (ql) edge[below] node {$\eps, \$ \to \eps$} (qf);
      }
    \end{tikzpicture}
    \caption{PDA for $L_6$}
  \end{figure}

  \step
  \textbf{Main Idea:}

  \step
  By definition, all strings $s \in L_6$ can be written as $s = yz$,
  where $y, z \in \set{0, 1}^*$, $\abs{y} = \abs{z}$, and $\beta(y) \equiv 0 \pmod 3$.

  Our PDA works in two general phases:
  \begin{itemize}
    \item Initially, we push a $\$$ symbol to the stack to mark
      the bottom of the stack. We then take an $\eps$-transition to $q_0$.
    \item The first phase of the PDA checks if the string processed
      (so far) is divisible by $3$. However, on every transition we also
      push a $\#$ symbol to the stack to track the number of symbols
      processed so far.
    \item At any time if we are at $q_0$ (indicating that the string processed
      so far is divisible by $3$), we can take an $\eps$-transition to $q_{loop}$
      without modifying the stack.
    \item While in $q_{loop}$, we read symbols from the alphabet and pop
      a single $\#$ symbol from the stack to loop back to $q_{loop}$.
      This process repeats until we reach the bottom of the stack,
      then we can pop the bottom-of-stack marker ($\$$) and transition
      to $q_{accept}$.
    \newpage
    \item We claim that the PDA accepts $L_6$.
      \begin{itemize}
        \item To reach $q_{accept}$ from $q_{start}$, we must make the transition
          from $q_0$ to $q_{loop}$. This means some prefix
          $x_1 x_2 \ldots x_n$ of the input string
          has a computational path that ends in $q_0$, meaning
          $\beta(x_1 x_2 \ldots x_n)$ is divisible y $3$.
        \item To reach $q_{accept}$ from $q_{loop}$, we must make the transition
          from $q_{loop}$ to $q_{accept}$ on reading $\eps$
          and popping $\$$ from the stack.
          This means that we have to pop all the $\#$'s from the stack
          at $q_{loop}$. Since the number of $\#$'s on the stack
          is equal to the length of $x_1 x_2 \ldots x_n$,
          we must process some string $y_1, y_2, \ldots, y_n$
          such that $\abs{y_1 y_2 \ldots y_n} = \abs{x_1 x_2 \ldots x_n}$

        \item Furthermore, $q_{accept}$ has no outgoing or looping transitions.
          For a computational path to end in $q_{accept}$,
          the last transition must be from $q_{loop}$ to $q_{accept}$,
          implying that $s = x_1 x_2 \ldots x_n y_1 y_2 \ldots y_n$.
          This means $s = yz$ for some $y, z \in \set{0, 1}^*$,
          $\abs{y} = \abs{z}$, and $\beta(y) \equiv 0 \pmod 3$,
          so $s \in L_6$.
      \end{itemize}
  \end{itemize}

\end{Answer}
