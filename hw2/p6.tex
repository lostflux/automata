\begin{problem}
  Recall the definition of $\textsc{Half}(L)$ from Homework $1$.
  Prove that if $L$ is regular, so is $\textsc{Half}(L)$.

  \[
    \textsc{Half}(L) = \set{x \in \Sigma^* : \exists y \in \Sigma^* \text{ such that } x y \in L}
  \]
  \begin{Answer}
    \step
    \textbf{High Level Idea}\\
    Since $L$ is regular, let $M = (Q, \Sigma, \delta, q_0, F)$
    be a DFA that recognizes $L$.
    We can track whether a string $s$ is a subsequence of another
    string $t$ in $L$ by tracking the set of states reachable from $q$
    that eventually reach an accepting state.
    We can start this tracking with the start state, $q_0$,
    and the set of accepting states, $F$,
    Each time we take a step in our DFA, we replace the set $S$ of tracked states
    by those states that have a transition into \emph{at least} one member of $S$.
    Consequently, every time we transition in our DFA,
    we also backtrack one step.
    If we reach a state after processing a string $s \in \textsc{Half}(L)$,
    then the state $q_s$ must be in the set of tracked states.

    \step
    \textbf{Formal Definition}\\
    Let $\psi : \calP(Q) \to \calP(Q)$ be a function as follows:
    \[ \psi(S) = \bigcup_{s \in S} \set{q \in Q : \exists y \in \Sigma, \delta(q, y) = s}.\]
    We may also define function exponentiation as repeated function application,
    i.e.
    \[
      \psi^n(S) = \begin{cases}
        S & \text{ if } n = 0 \\
        \psi^{n-1}(\psi(S)) &\text{ if } n \ge 0
      \end{cases}
    \]
  
    \step
    Define a new DFA $M_2$ as follows:
    \begin{align*}
      M_2 &= (Q \times \calP(Q), \Sigma, \delta_2, q_{0_2}, F_2) \text{ where }\\
      q_{0_2} &= (q_0, F) \\
      \delta_2 \left((q, S), x \right) &= \left(\delta(q, x), \psi(S) \right) \\
      F_2 &= \set{(q, S) \in Q \times \calP(Q) : q \in S }.
    \end{align*}

    \begin{claim}~\label{claim:half}
      $\calL(M_2) = \textsc{Half}(L)$.
      \begin{proof}

        \step
        \begin{enumroman}
          \item \textbf{Completeness} ($\calL(M_2) \supseteq \textsc{Half}(L)$)
          
          \step
          Suppose a string $u$ is a member of $\textsc{Half}(L)$, then
          there exists a string $t \in L$ and a string $v \in \Sigma^*$
          such that $\abs{u} = \abs{v}$ and $t = uv$.
          Let $t = t_1, \ldots, t_n, \ldots, t_{2n}$ where $t_i \in \Sigma$ for all $0 < i \le 2n$, and
          let $p = p_0, \ldots, p_n, \ldots p_{2n}$ be the computational path of $M$ on $t$
          such that $p_0 = q_0$ and $p_i = \delta(p_{i-1}, t_i)$.
          Since $t \in L$, $M$ recognizes $t$, $p_n \in F$.
          Furthermore, $t = uv$ and $\abs{u} = \abs{v}$, so we can write
          $u = t_1, \ldots, t_n$ and $v = t_{n+1}, \ldots, t_{2n}$.
          
          \step
          Consider the computational path of $M_2$ on $u$.
          $M_2$ starts in state $(q_0, F)$ and processes each of the characters
          $t_1$ up to $t_n$ in sequence, transitioning from the states
          $(q_0, F)$ to $(q_1, \psi(F))$, to $(q_2, \psi^2(F))$, all the way to $(q_n, \psi^n(F))$ and so on.

          \step
          Recall that $t \in L$ and $p = p_1 \ldots p_n, \ldots p_{2n}$ is the computational
          path of $M$ on $t$.
          Therefore, $p_{2n} \in F$.
          Since $\delta(p_{2n-1}, t_n) = p_{2n}$, we have that
          $p_{2n-1} \in \psi(F)$. Similarly, $\delta(p_{2n-2}, t_{2n-1}) = p_{2n-1}$,
          so $p_{2n-2} \in \psi^2(F)$,
          and $p_{2n-i} \in \psi^i(F)$ for all $0 < i \le n$.
          Therefore, $p_n \in \psi^n(F)$, and $(q_n, \psi^n(F))$ is an accepting state of $M_2$.

          \item \textbf{Soundness} ($\calL(M_2) \subseteq \textsc{Half}(L)$)
            
          \step
          Suppose a string $s$ is accepted by $M_2$.
          Let $s = s_1, \ldots, s_n$ where $s_i \in \Sigma$ for all $0 < i \le n$.
          Let $p = p_0, \ldots, p_n$ be the computational path of $M_2$ on $s$,
          then $p_n \in \psi^n(F)$ (by definition of the accepting states of $M_2$).
          That implies that there exists a state $p_{n+1} \in Q$ such that
          $\delta(p_n, x) = p_{n+1}$ for some $x \in \Sigma$.
          Therefore, $p_{n+1} \in \psi^{n-1}(F)$.
          Similarly, there must exist a state $p_{n+2} \in Q$ such that
          $\delta(p_{n+1}, x_1) = p_{n+2}$ for some $x_1 \in \Sigma$.
          This implies again that $p_{n+2} \in \psi^{n-2}(F)$,
          so there must exist a state $p_{n+3} \in Q$, such that
          $\delta(q_{n+2}, x_3) = p_{n+3}$ for some $x_3 \in \Sigma$.
          More generally, for each $0 < i \le n$, there must exist a state $p_{n+i} \in Q$ such that
          $\delta(p_{n+i-1}, x) = p_{n+i}$ for some $x \in \Sigma$.
          Since it took $n$ transitions to reach the accepting state $(p_n, \psi^n(F))$,
          there must be $n$ states $p_{n+1}, \ldots, p_{2n}$ such that
          $\delta(p_{n+i-1}, x) = p_{n+i}$ for all $0 < i \le n$,
          and $p_{2n} \in F$, meaning
          the corresponding string $s' = s_1 \ldots s_{2n}$ is a member of $L$.
          Therefore, $s \in \calL(M_2)$.
        \end{enumroman}
      \end{proof}
    \end{claim}
  \end{Answer}
\end{problem}
