\begin{problem}
  Recall the definitions of $\textsc{Max}(L)$ and $\textsc{Min}(L)$ from Homework $1$.
  Prove that, for every regular language $L$, the languages
  $\textsc{Max}(L)$ and $\textsc{Min}(L)$ are both regular.

  \step
  \begin{align*}
    \textsc{Min}(L) &= \set{x \in \Sigma^* : x \in L \text{ and no proper prefix of } x \in L},\\
    \textsc{Max}(L) &= \set{x \in \Sigma^* : x \in L \text{ and $x$ is not a proper prefix of any string in $L$}}.
  \end{align*}
  \begin{Answer}
    Since $L$ is regular, there exists a DFA M that recognizes $L$.
    Let $M = (Q, \Sigma, \delta, q_0, F)$ be such a DFA.
    \begin{enumalph}
      \item $\textsc{Min}(L)$
        
        \step
        Note that $\textsc{Min}(L) \subseteq L$.
        Let \zaff{$s = s_1\ldots s_n, n \ge 1, s_i \in \Sigma$} be a string in $L$
        such that $s \notin \textsc{Min}(L)$,
        then $\exists\; \zaff{t = s_1 \ldots s_k, k < n}$, such that $t \in L$.
        Let $p_s = p_0,\ldots, p_n$ be the computation path of M on $s$
        such that $p_0 = q_0$ and $p_i = \delta(p_{i-1}, s_i)$ for all $i$, and $p_n \in F$.
        Then $p_t$, the computation path of M on $t$,
        is equivalent to $p_0, \ldots, p_k$ for some $k < n$
        since the first $k$ letters in $s$ are identical to the first $k$ letters in $t$
        and both computation paths start at $q_0$ yet DFAs may only transition
        to a single state given an input letter and a state.
        Therefore, the computation path of M on a string $s \notin \textsc{Min}(L)$
        always enters and leaves
        at least one accepting state before reaching the final state.
        To prune $s$, redirect all transitions \emph{out of} accepting states
        to a new trap-state $r$ that does not accept any strings.
        \begin{align*}
          M_2 &= (Q \cup \set{r}, \Sigma, \delta_2, q_0, F) \text{ where }\\
          r &\notin Q \\
          \delta_2(q, a) &= \begin{cases}
            r & \text{if $q \in F \cup \set{r}$ } \\
            \delta(q, a) & \text{otherwise } \\
          \end{cases}
        \end{align*}
        \begin{enumroman}
          \item $\calL(M_2) \supseteq \textsc{Min}(L)$ (Completeness).
            
            Suppose a string $s$ is in $\textsc[Min]{L}$,
            then $s$ must not have a proper prefix that happens to be in $L$.
            Therefore, the computation path of $M_2$ on $s$ never enters \emph{and leaves} an accepting state,
            and $M_2$ does identical transitions to those on $M$ on $s$
            since $\delta_2(q, x) = \delta(q, x)$ for all $q \in Q \setminus F$ and $x \in \Sigma$.
            Since $s \in L$, M accepts $s$, so $M_2$ also accepts $s$.
          \item $\calL(M_2) \subseteq \textsc{Min}(L)$ (Soundness).
          
            Suppose a string $s$ is accepted by $M_2$,
            then the computation path of $M_2$ on $s$ never enters \emph{and leaves} an accepting state,
            since $\delta_2(q, x) = r$ for all $q \in F$ and $x \in \Sigma$.
            Therefore, $s$ must not have a proper prefix that happens to be in $L$.
        \end{enumroman}
      \item $\textsc{Max}(L)$
      
      \step
      Let $M$ be a DFA that recognizes $L$.

      We define:
      \begin{align*}
        \delta^* : Q \times \Sigma^* &\to Q \\
        \delta^*(q, x) &= \begin{cases}
          q &\text{ if $x = \eps$. }\\
          \delta^*(\delta(q, s), t) &\text{ otherwise, where $q = st, s \in \Sigma, t \in \Sigma^*$. } \\
        \end{cases}\\
        R &= \set{q \in F : \delta^*(q, s) \in F \text{ for some } s \in \Sigma^* \setminus \set{\eps}}.
      \end{align*}
      \begin{observation}
        $R$ is the set of all \emph{accepting states} in $M$
        and have some non-empty transition sequence to some accepting state
        (which could be the same state).
      \end{observation}

      \step
      Let $M_2$ be defined as follows:
      \[ M_2 = (Q, \Sigma, \delta, q_0, F \setminus R)\]
      \begin{claim}~\label{claim:Max}
        $\calL(M_2) = \textsc{Max}(L)$
        \begin{proof}

          \step
          \begin{enumroman}
            \item \textbf{Completeness} ($\calL(M_2) \supseteq \textsc{Max}(L)$)
            Suppose a string $s$ is in $\textsc{Max}(L)$,
            then $s \in L$, since $\textsc{Max}(L) \subseteq L$.
            Let $s = s_1 \ldots s_n, n \ge 1, s_i \in \Sigma$,
            and let $p = p_0, \ldots, p_n$ be the computation path of $M$ on $s$
            where $p_0 = q_0$ and $p_i = \delta(p_{i-1}, s_i)$ for all $i$.
            Then $p_n \in F$ since $s \in L$.
            
            \step
            However, $s$ is not a proper substring of any string in $L$
            (since $s \in \textsc{Max}(L)$), therefore $p_n \notin R$.
            This implies that $p_n \in F \setminus R$.
            Since $M_2$ uses the same transition function as $M$,
            the computation path of $M_2$ on $s$ is equivalent to $p_0, \ldots, p_n$
            and $p_n$ is an accepting state in $M_2$ so $M_2$ accepts $s$.
            \item \textbf{Soundness} ($\calL(M_2) \subseteq \textsc{Max}(L)$)
              Let $s$ be a string accepted by $M_2$.
              Let $s = s_1 \ldots s_n, n \ge 1, s_i \in \Sigma$,
              and $p = p_0, \ldots, p_n$ be the computation path of $M_2$ on $s$
              where $p_0 = q_0$ and $p_i = \delta(p_{i-1}, s_i)$ for all $i$.
              Then $p_n \in F \setminus R$ since $M_2$ accepts $s$.
              This implies two things:
              \begin{itemize}
                \item $p_n \in F$, meaning $M$, the original DFA, accepts $s$.
                  Therefore, $s$ is a member of the language $L$.
                \item $p_n \notin R$, meaning $p_n$ does not have a non-empty transition sequence to some accepting state.
                  Therefore, there does not exist any non-empty string $y \in \Sigma^*$
                  such that $\delta^*(p_n, y) \in F$.
                  This implies that there is no string in $\Sigma^8$ that, when appended to $s$,
                  results in a string that is accepted by $M$ (that is, a string in $L$).
                  This is the same as saying that $s$ is not a proper substring of any string in $L$,
                  so $s \in \textsc{Max}(L)$.
              \end{itemize}
          \end{enumroman}
        \end{proof}
      \end{claim}
    \end{enumalph}
  \end{Answer}
\end{problem}
