\begin{problem}
  Recall the definitions of $\textsc{Max}(L)$ and $\textsc{Min}(L)$ from Homework $1$.
  Prove that, for every regular language $L$, the languages
  $\textsc{Max}(L)$ and $\textsc{Min}(L)$ are both regular.

  \step
  \begin{align*}
    \textsc{Min}(L) &= \set{x \in \Sigma^* : x \in L \text{ and no proper prefix of } x \in L},\\
    \textsc{Max}(L) &= \set{x \in \Sigma^* : x \in L \text{ and $x$ is not a proper prefix of any string in $L$}}.
  \end{align*}
  \begin{Answer}
    Since $L$ is regular, there exists a DFA M that recognizes $L$.
    \begin{enumalph}
      \newpage
      \item $\textsc{Min}(L)$
      
      \step
        Let $M = (Q, \Sigma, \delta, q_0, F)$ be a DFA recognizing $L$.
        Let $s = s_1\ldots s_n, n \ge 1, s_i \in \Sigma$ be a string in $L$
        such that \crim{$s \notin \textsc{Min}(L)$},
        then there exists some string $t \in L$ that is a proper substring of $s$.
        That is, $t = s_1 \ldots s_k, k < n$.
        Consider the computational path of $M$ on $s$,
        $p_s = p_0,\ldots, p_n$ such that $p_0 = q_0$,
        $p_i = \delta(p_{i-1}, s_i)$ for all $i$, and $p_n \in F$.
        Then $p_t$, the computation path of M on $t$,
        is equivalent to $p_0, \ldots, p_k$
        since the first $k$ letters in $s$ are identical to the first $k$ letters of $t$
        and both computation paths start at the same state, $q_0$, yet DFAs may only transition
        to a single state given an input letter and a state.
        Since $t \in L$, $p_t$ must end in an accepting state,
        so the computation path of $M$ on $s \notin \textsc{Min}(L)$
        \emph{must} enter and leave an accepting state.
        To recognize members of $L$ that are in $\textsc{Min}(L)$,
        we can prune those that are not in $\textsc{Min}(L)$ by redirecting
        all transitions out of accepting states to a new trap-state $r$
        that rejects.
        \begin{align*}
          M_2 &= (Q \cup \set{r}, \Sigma, \delta_2, q_0, F) \text{ where $r \notin Q$ and }\\
          \delta_2(q, x) &= \begin{cases}
            r & \text{if $q \in F \cup \set{r}$ } \\
            \delta(q, x) & \text{otherwise } \\
          \end{cases}
        \end{align*}
        \begin{enumroman}
          \item \textbf{Completeness} ($\calL(M_2) \supseteq \textsc{Min}(L)$)
            
            Let $s$ be a string such that $s \in \textsc{Min}{L}$,
            then $s$ does not have a proper prefix that happens to be in $L$.
            Let $p_s = p_0, \ldots, p_n$ be the computation path of $M$ on $s$,
            such that $p_0 = q_0$ and $p_i = \delta(p_{i-1}, s_i)$ for all $i$.
            Then $p_i \notin F$ for all $i<n$, so $\delta_2(p_i, x) = \delta(p_i, x)$.
            Particularly, $M_2$ never transitions to $r$, since the state $r$ 
            does not exist in the original DFA, $M$, yet $M_2$ is using the same transition
            function as $M$ when $p-i \notin F$.
            Therefore, $M_2$ never enters \emph{and leaves} an accepting state,
            but eventually transitions into an accepting state $p_n$, thereby accepting $s$.
            \item \textbf{Soundness} ($\calL(M_2) \subseteq \textsc{Min}(L)$)
          
            Let $s$ be a string accepted by $s$.
            Let $p_s = p_0, \ldots, p_n$ be the computation path of $M_2$ on $s$,
            then $p_i \ne r$ for all $i$ since $r$ is a rejecting trap-state.
            Notice that the transition function, $\delta_2$, always transitions
            to $r$ when leaving a state in $F$, so none of the states that
            $M_2$ transitions out of may be in $F$.
            But $F$ is the same set of states that $M$, the original DFA, accepts!
            Therefore, none of the states substrings in $s$ are accepted by $M$,
            implying that they are not in $L$.
            Consequently, $s$ is a member of $L$, and no proper substring
            of $s$ is in $L$, so $s$ must be in $\textsc{Min}(L)$.
        \end{enumroman}
      \newpage
      \item $\textsc{Max}(L)$
      
      \step
      Let $M$ be a DFA that recognizes $L$.

      We define:
      \begin{align*}
        \delta^* : Q \times \Sigma^* &\to Q \\
        \delta^*(q, x) &= \begin{cases}
          q &\text{ if $x = \eps$. }\\
          \delta^*(\delta(q, s), t) &\text{ otherwise, where $q = st, s \in \Sigma, t \in \Sigma^*$. } \\
        \end{cases}\\
        R &= \set{q \in F : \delta^*(q, s) \in F \text{ for some } s \in \Sigma^* \setminus \set{\eps}}.
      \end{align*}
      \begin{observation}
        $R$ is the set of all \emph{accepting states} in $M$
        that have some non-empty transition sequence to some other accepting state
        (which could be the same state).
      \end{observation}

      \step
      Let $M_2$ be defined as follows:
      \[ M_2 = (Q, \Sigma, \delta, q_0, F \setminus R)\]
      \begin{claim}~\label{claim:Max}
        $\calL(M_2) = \textsc{Max}(L)$
        \begin{proof}

          \step
          \begin{enumroman}
            \item \textbf{Completeness} ($\calL(M_2) \supseteq \textsc{Max}(L)$)
            
            \step
            Suppose a string $s$ is in $\textsc{Max}(L)$,
            then $s \in L$, since $\textsc{Max}(L) \subseteq L$.
            Let $s = s_1 \ldots s_n, n \ge 1, s_i \in \Sigma$,
            and let $p = p_0, \ldots, p_n$ be the computation path of $M$ on $s$
            where $p_0 = q_0$ and $p_i = \delta(p_{i-1}, s_i)$ for all $i$.
            Then $p_n \in F$ since $s \in L$.
            
            \step
            However, $s$ is not a proper substring of any string in $L$
            (since $s \in \textsc{Max}(L)$), therefore $p_n \notin R$.
            This implies that $p_n \in F \setminus R$.
            Since $M_2$ uses the same transition function as $M$,
            the computation path of $M_2$ on $s$ is equivalent to $p_0, \ldots, p_n$
            and $p_n$ is an accepting state in $M_2$ so $M_2$ accepts $s$.
            \item \textbf{Soundness} ($\calL(M_2) \subseteq \textsc{Max}(L)$)
              
            
            \step
            Let $s$ be a string accepted by $M_2$.
            Let $s = s_1 \ldots s_n, n \ge 1, s_i \in \Sigma$,
            and $p = p_0, \ldots, p_n$ be the computation path of $M_2$ on $s$
            where $p_0 = q_0$ and $p_i = \delta(p_{i-1}, s_i)$ for all $i$.
            Then $p_n \in F \setminus R$ since $M_2$ accepts $s$.
            This implies two things:
            \begin{itemize}
              \item $p_n \in F$, meaning $M$, the original DFA, accepts $s$.
                Therefore, $s$ is a member of the language $L$.
              \item $p_n \notin R$, meaning $p_n$ does not have a non-empty transition sequence to some accepting state.
                Therefore, there does not exist any non-empty string $y \in \Sigma^*$
                such that $\delta^*(p_n, y) \in F$.
                This implies that there is no string in $\Sigma^8$ that, when appended to $s$,
                results in a string that is accepted by $M$ (that is, a string in $L$).
                This is the same as saying that $s$ is not a proper substring of any string in $L$,
                so $s \in \textsc{Max}(L)$.
            \end{itemize}
          \end{enumroman}
        \end{proof}
      \end{claim}
    \end{enumalph}
  \end{Answer}
\end{problem}
