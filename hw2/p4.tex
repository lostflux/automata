\begin{problem}
  \begin{enumalph}
    \item Let $L$ be a nonempty language recognized by an NFA $N$.
      Prove that there exists an NFA $N_1$ that has exactly one accept state
      and recognizes the same language $L$.
      \begin{Answer}
        Let $N = (Q, \Sigma, \delta, q_0, F)$ be an NFA that recognizes $L$.
        
        \step
        We define $N_1 = (Q \cup \set{a}, \Sigma, \delta_1, q_0, F_1)$,
        where $a$ is a new state that is not in $Q$, and:
        \begin{align*}
          F_1 &= \set{a} \\
          \delta_1(q, x) &= \begin{cases}
            \delta(q, x) \cup \set{a} &\text{ if $q \in F$ and $x = \eps$ }\\
            % \set{a} &\text{ if $q = a$ }\\
            \delta(q, x) & \text{ otherwise. }
          \end{cases}
        \end{align*}
        \begin{claim}
          $N_1$ is an NFA that recognizes $L$.
          \begin{proof}
            
            \step
            \begin{enumroman}
              \item \textbf{Completeness} ($\calL(N_1) \supseteq L$)

                \step
                Let $w = w_1\ldots w_k$ be a string in $L$
                such that $w_i \in \Sigma$ for all $0 < i \le k$.
                Since $N$ recognizes $L$,
                there exists a computation path $p_w = p_0,\ldots,p_n$
                of $N$ on $w$ such that $p_n \in F$.
                Note that the state $a$ does not exist in $N$, the original NFA,
                so $p_i \neq a$ for all $0 \le i \le n$.
                Consider the computation path of $N_1$ on $w$.
                Since $\delta_1(q, x) \supseteq \delta(q, x)$
                for all states $q \in Q \setminus \set{a}$,
                the computation path of $N$ on $w$ is always one of the possible
                computation paths of $N_1$ on $w$.
                Suppose $N_1$ chooses that computation path, then it will eventually
                transition to the state $p_n$, and $p_n \in F$.
                There, $N_1$ will have the option of taking an epsilon-transition
                to $a$, the single accepting state in $N_1$.
                $N_1$ can therefore transition to $a$ and accept $w$
                whenever $w$ is in $L$.
                
              \item \textbf{Soundness} ($\calL(N_1) \subseteq L$)
              Suppose $N_1$ accepts a string $w = w_1\ldots w_n$, $n \ge 1$, $w_i \in \Sigma \cup \set{\eps}$
              for all $1 \le i \le n$.
              Let the computation path of $N_1$ on $w$ be $p = p_0,\ldots, p_n$
              such that $p_0 = q_0$ and $p_i = \delta_1(p_{i-1}, w_i)$ for all $1 \le i \le n$.
              Since $N_1$ has exactly one accept state, $p_n = a$.
              However, there is only one epsilon-transition into $a$,
              which happens from a state $q \in F$.
              We can also observe that there are no transitions from $a$,
              so any path that ends at $a$ must have transitioned into $a$
              at the last step. Therefore, $p_{n-1} \in F$.
              Since no transitions into $a$ were taken before the last step,
              all transitions were identical to those that would be taken by $N$
              on $w$, meaning the computational path of $N$ on $w$ would also terminate
              at $p_{n-1}$, so $N$ accepts $w$.
              Therefore, any sequence $w = w_1\ldots w_{n}$ accepted by $N_2$
              is also accepted by $N$, implying that $w \in L$.
            \end{enumroman}
          \end{proof}
        \end{claim}
        
      \end{Answer}

    % \newpage
    \item Prove, by giving a concrete counterexample,
      that the analogous result does not hold for DFAs,
      i.e., that for a nonempty language $L$ recognized by a DFA $M$,
      there might not exist any $1$-accept-state DFA that recognizes $L$.
      Explain clearly why your chosen language $L$ has this property.
      \begin{Answer}
        For the alphabet $\Sigma = \set{0, 1}$,
        let $L = \set{x \in \Sigma^* : \abs{x} \in \set{2, 3}}$
        Consider the DFA that recognizes $L$:
        \begin{align*}
          M &= (Q, \Sigma, \delta, q_0, F) \quad \text{ where } \quad \\
          Q &= \set{q_i : 0 \le i \le 4} \\
          \delta(q_i, x) &= q_{i + 1}\\
          q_0 &= q_0 \\
          F &= \set{q_2, q_3}
        \end{align*}

        \step
        Note that the DFA has two accepting states, $q_2$ and $q_3$.
        However, we have that $\delta(q_2, x) = q_3$ but $\delta(q_3, x) = q_4$,
        i.e. $M$ transitions from $q_2$ to an accepting state ($q_3$),
        but $M$ transitions from $q_3$ to a non-accepting state ($q_4$).
        Therefore, we cannot merge $q_2$ and $q_3$ into a single state.
        Furthermore, a DFA cannot move to another state without getting any
        valid input symbol $x \in \Sigma$, so we cannot add epsilon-transitions
        from $q_2$ and $q_3$ into a new accepting state.

        \step
        We therefore \emph{have to} retain $q_2$ and $q_3$ as separate states
        in order to recognize $L$.
      \end{Answer}
  \end{enumalph}
\end{problem}
