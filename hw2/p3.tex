\begin{problem}
  Let $L$ be the language over $\Sigma = \set{a, b}^*$ given by the regular expression
  $(ab \; \cup \; aab \; \cup \; aba)^*$.
  \begin{enumalph}
    \item Design an NFA for $L$ that has exactly $4$ states,
      no $\eps$-transitions, and exactly one $a$-transition out of its start state.
    \begin{Answer}
      % Draw the NFA
      \begin{figure}[H]
        \centering
        \begin{tikzpicture}
          \node[state, initial, accepting] (q) at (0, 0) {$q$};
          \node[state, right of=q] (r) {$r$};
          \node[state, above of=r] (s) {$s$};
          \node[state, below of=r] (t) {$t$};
          \draw (q) edge [bend left, above] node {$a$} (r)
                (r) edge [bend left, below] node {$b$} (q)
                (r) edge [right] node {$a$} (s)
                (r) edge [right] node {$b$} (t)
                (s) edge [bend right, above] node {$b$} (q)
                (t) edge [bend left, below] node {$a$} (q);
        \end{tikzpicture}
        \caption{NFA to recognize $L \equiv (ab \; \cup \; aab \; \cup \; aba)^*$}
        \label{fig:3a}
      \end{figure}
    \end{Answer}
    
    \newpage
    \item Convert the above NFA into a DFA for $L$
      by mechanically using the power set construction.
      For the sake of legibility, may avoid drawing transitions
      out of states that are unreachable from the start state of the resulting DFA.
      However, do draw every state of the power set construction.
      \emph{I strongly recommend using state names like $rt$, $qst$, etc. in your
      drawing and not the more formal $\set{r, t}, \set{q, s, t}$, etc.}
    \begin{Answer}
      % Draw the NFA
      \begin{figure}[H]
        \centering
        \begin{tikzpicture}
          \node[state, initial, accepting] at (0, 0) (q) {$q$};
          \node[state, below of=q] (r) {$r$};
          \node[state, left of=r] (null) {$\emptyset$};
          \node[state, below of=r] (s) {$s$};
          \node[state, below of=s] (t) {$t$};
          \node[state, right of=q, accepting] (qr) {$qr$};
          \node[state, below of=qr, accepting] (qt) {$qt$};
          \node[state, below of=qt] (rs) {$rs$};
          \node[state, below of=rs, accepting] (qs) {$qs$};
          \node[state, below of=qs] (rt) {$rt$};
          \node[state, below of=rt] (st) {$st$};
          \node[state, right of=qr, accepting] (qrs) {$qrs$};
          \node[state, below of=qrs, accepting] (qrt) {$qrt$};
          \node[state, below of=qrt, accepting] (qst) {$qst$};
          \node[state, below of=qst] (rst) {$rst$};
          \node[state, right of=qrs, accepting] (qrst) {$qrst$};

          \draw (q) edge[left] node {$a$} (r)
                (q) edge[left] node {$b$} (null)

                (r) edge[left] node {$a$} (s)
                (r) edge[above] node {$b$} (qt)

                (s) edge[left, bend left] node {$b$} (q)
                (s) edge[left] node {$a$} (null)

                (qt) edge[left, bend left] node {$a$} (qr)
                (qt) edge[below, bend left, pos=0.2] node {$b$} (null)

                (qr) edge[right, bend left, pos=0.85] node {$a$} (rs)
                (qr) edge[left, bend left, pos=0.75] node {$b$} (qt)
                
                (rs) edge[below] node {$a$} (s)
                (rs) edge[left] node {$b$} (qt);
        \end{tikzpicture}
        \caption{Seven-state DFA to recognize $L \equiv (ab \; \cup \; aab \; \cup \; aba)^*$}
        \label{fig:3b}
      \end{figure}
    \end{Answer}


    \item Observe that exactly $7$ states are reachable,
      so if you were to delete the unreachable ones,
      you would obtain a $7$-state DFA for $L$.
      \emph{Carefully observe this DFA and argue that two of its states
      can be replaced by a single state.}
      Do this and draw the resulting $6$-state DFA for $L$.
    \begin{Answer}
      The two states $r$ and $rs$ can be merged because the y
      are both non-accepting states and the
      transitions out of the two states are identical,
      i.e. $\delta(r, x) = \delta(rs, x)$ for all $x \in \Sigma$.

      \step
      Resulting DFA:
      % Draw the DFA
      \begin{figure}[H]
        \centering
        \begin{tikzpicture}
          \node[state, initial, accepting] at (0, 0) (q) {$q$};
          \node[state, below of=q] (rs) {$rs$};
          \node[state, left of=r] (null) {$\emptyset$};
          \node[state, below of=r] (s) {$s$};
          \node[state, right of=q, accepting] (qr) {$qr$};
          \node[state, below of=qr, accepting] (qt) {$qt$};

          \draw (q) edge[left] node {$a$} (rs)
                (q) edge[left] node {$b$} (null)

                (s) edge[left, bend left] node {$b$} (q)
                (s) edge[left] node {$a$} (null)

                (qt) edge[left, bend left] node {$a$} (qr)
                (qt) edge[below, bend left, pos=0.2] node {$b$} (null)

                (qr) edge[right, pos=0.85] node {$a$} (rs)
                (qr) edge[left, bend left, pos=0.75] node {$b$} (qt)
                
                (rs) edge[right] node {$a$} (s)
                (rs) edge[above, pos=0.8] node {$b$} (qt);
        \end{tikzpicture}
        \caption{Six-state DFA to recognize $L \equiv (ab \; \cup \; aab \; \cup \; aba)^*$}
        \label{fig:3c}
      \end{figure}
    \end{Answer}

    \newpage
    \item Give clear reasons why $L$ cannot be recognized by a DFA with 5 or fewer states.
    For extra credit, write your reasoning as a formal proof.

    \step
    \grey{
      \emph{
        Hint: Argue that any hypothetical DFA for $L$ must treat the strings
        $aa$ and $ab$ differently in the sense that the state it reaches upon reading $aa$
        must be different from the state it reaches upon reading $ab$.
        Now extend this observation by identifying $6$ specific strings
        (which may not all belong to $L$) that must all be treated differently.
      }
    }
    \begin{Answer}
      Any DFA that recognizes $L$ must differentiate between $6$
      unique strings, each of which may or may not belong to $L$
      but has a unique set of $\delta$-transitions.
      These strings are:
      \begin{enumroman}
        \item $aa$ and $ab$:
          The string $aa$ does not belong to $L$, but the string $ab$ does.
          Any DFA that recognizes $L$ must therefore treat these two strings differently.
        \item $aab$ and $aba$:
          While both $aab$ and $aba$ belong in $L$,
          note that upon reading a $aab$ followed by $b$,
          any DFA that recognizes $L$ must transition to a reject state
          (since $aabb \notin L$)
          while the same DFA must transition to an accept state
          upon reading $aba$ followed by $b$
          (since $abab \in L$). Therefore, the DFA may not transition to
          the same state upon reading $aab$ vs. $aba$, since
          its transitions from that state must be distinct from each other.
        \item $aaab$ and $abab$:
          The string $aaab$ does not belong to $L$, but the string $abab$ does.
          Any DFA that recognizes $L$ must therefore treat these two strings differently.
      \end{enumroman}
    \end{Answer}
  \end{enumalph}
\end{problem}
