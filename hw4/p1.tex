\begin{problem}
  Draw a PDA that recognizes the language
  $L = \set{x \in \set{0, 1}^* : N_1(x) \ge 2N_0(x)}$.
  Give a high-level proof that your that your PDA works
  correctly.
\end{problem}
\begin{Answer}
  % Draw the PDA
  \begin{figure}[H]
    \begin{tikzpicture}
      \node[state, initial] (q0) at (0, 0) {$q_0$};
      \node[state] at (5, 0) (q1) {$q_1$};
      \node[state] at (10, 0) (q2) {$q_2$};
      \node[state, below of=q2, accepting] (q3) {$q_3$};
      \draw (q0) edge[above, pos=0.3] node {$\eps, \eps \to \$$} (q1)
            (q1) edge[loop above] node {$
            \begin{aligned}
              1, p &\to pp \\
              1, m &\to \eps\\
              1, \$ &\to p\$
            \end{aligned}$ } (q1)
            (q1) edge[loop below] node {$
            \begin{aligned}
              0, m \to mmm\\
              0, \$ \to mm\$
            \end{aligned}$ } (q1)
            (q1) edge[below] node {$0, p \to \eps$} (q2)
            (q2) edge[bend right, above, pos=0.3] node {$
            \begin{aligned}
              \eps, p \to \eps\\
              \eps, \$ \to m\$
            \end{aligned}$} (q1)
            (q1) edge[below, pos=0.7] node {$
            \begin{aligned}
              \eps, p \to \eps\\
              \eps, \$ \to \eps
            \end{aligned}$} (q3);
    \end{tikzpicture}
    \caption{A PDA recognizing $L$}
  \end{figure}

  \newpage
  \step
  \textbf{High-Level Idea and Proof of Correctness}

  \step
  We use have stack variables: $p$, $m$, and $\$$.
  Using these, we track the value of $N_1(x) - 2N_0(x)$
  as we read the string. A $p$ corresponds to a `$+1$', an $m$ corresponds
  to a `$-1$', and $\$$ corresponds to a $0$.
  This is how the PDA works:
  \begin{itemize}
    \item First, we enforce that no stack state can contain both $p$'s and $m$'s
      at the same time. We do this by only starting to push $p$'s (or $m$'s
      in the alternate case) if the symbol at the top of the stack is $\$$,
      symbolizing a $0$.
    \item We start by pushing a $\$$ onto the stack, signifying a state of $0$.
    \item Whenever we read a $0$, we decrease the stack state by $2$.
      This takes three forms:
      \begin{itemize}
        \item We can remove two $p$'s from the stack.
        \item If we only have a single $p$ at the top of the tack, we
          remove it and push a single $m$.
        \item If we have an $m$ or the zero marker ($\$$) at the top of the stack,
          we return it and push two more $m$'s.
      \end{itemize}
    \item Whenever we read a $1$, we increase the stack state by $1$.
      We do this by:
      \begin{itemize}
        \item If we have a $p$ or a $\$$ at the top of the stack,
          return it and push another $p$.
        \item If we have an $m$ at the top of the stack, remove it.
      \end{itemize}
    \item Consequently, when we reach the end of the string:
      \begin{itemize}
        \item If we have a $\$$ at the top of the stack,
          that means we have encountered \emph{exactly} twice as many $1$'s as $0$'s,
          so we accept the string.
        \item If we have a $p$ at the top of the stack,
          that means the number of $0$'s we have encountered is
          more than twice the number of $1$'s we have encountered,
          so we accept the string.
        \item Otherwise, the number of $1$'s was less than twice the number
          $0$'s in the string, so we do not generate a transition to the
          accepting state.
      \end{itemize}
  \end{itemize}
\end{Answer}
