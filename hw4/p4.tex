\begin{problem}
  A string $x \in \Sigma^*$ is called a \emph{square}
  if $x = w^2$ for some $w \in \Sigma^*$.
  Let $L_{sq} = \set{w^2 : w \in \set{0, 1}^*}$.
  Consider its complement:
  \[ \overline{L}_{sq} =
    \set{x \in \set{0, 1}^* : x
      \text{
        is not of the form $w^2$ for any $w \in \set{0, 1}^*$
      }
    }.
  \]

  \step
  \begin{enumalph}
    \item Prove that every even-length string is in $\overline{L}_{sq}$
      can be decomposed as $x = uv$ where the middle symbol of $u$
      differs from the middle symbol of $v$.
      \begin{Answer}
        Let $x = uv$ be a string in $\overline{L}_{sq}$
        such that $\abs{u} = \abs{v}$.
        Suppose the string $x$ has length $2n$, such that
        $x = u_1u_2 \cdots u_n v_1v_2 \cdots v_n$.
        Since $u \ne v$ (by definition of $\overline{L}_{sq}$),
        it must be the case that $u_i \ne v_i$ for some $1 \le i \le n$
        (maybe multiple values of $i$, but we only care about one).
        
        \step
        Suppose $k$ is the smallest such $i$ with $u_k \ne v_k$.
        \begin{enumroman}
          \item If $k \le \frac{n}{2}$, take
            $s_1 = u_1, u_2, \ldots, u_{2k-1}, \ldots, u_n$
            and $s_2 = v_1, v_2, \ldots, v_{2k-1}, \ldots, v_n$.
            Then $\abs{s_1} = 2k-1$ meaning the middle symbol of $s_1$
            is $u_{k}$. Likewise, $\abs{s_2} = n + n-(2k - 1) = 2n - 2k + 1$,
            meaning the middle element is at position $n - k$.
            Since $s_1$ starts at $2k$, the middle element is at position
            $2k + n - k = n + k$.
            This is the element corresponding to $v_k$ and $u_k \ne v_k$
            so the two strings $s_1$ and $s_2$ have differing middle symbols.
          \item If $k > \frac{n}{2}$, proceed as above but counting up to the
            corresponding element in $v$ from the end of the string.
        \end{enumroman}
      \end{Answer}
    \item Using this property, design a context-free grammar that
      generates $\overline{L}_{sq}$.
      \begin{Answer}
        \textbf{Derivations}
        \begin{align*}
          S &\derives AB \mid BA \mid X \\
          A &\derives 0A0 \mid 0A1 \mid 1A0 \mid 1A1 \mid 0 \\
          B &\derives 0B0 \mid 0B1 \mid 1B0 \mid 1B1 \mid 1 \\
          X &\derives 0X0 \mid 0X1 \mid 1X0 \mid 1X1 \mid 0 \mid 1
        \end{align*}
        \textbf{CFG}
        \begin{align*}
          G &= (V, \Sigma, R, S) \quad \text{ where: } \\
          V &= \set{S, A, B, X} \\
          \Sigma &= \set{0, 1} \\
          R &= R_S \cup R_A \cup R_B \cup R_X \\
          &R_S = \set{(S, AB), (S, BA), (S, X)} \\
          &R_A = \set{(A, 0A0), (A, 0A1), (A, 1A0), (A, 1A1), (A, 0)} \\
          &R_B = \set{(B, 0B0), (B, 0B1), (B, 1B0), (B, 1B1), (B, 1)} \\
          &R_X = \set{(X, 0X0), (X, 0X1), (X, 1X0), (X, 1X1), (X, 0), (X, 1)}
        \end{align*}

        \step
        \textbf{Idea Behind CFG}

        \step
        First, note that odd-length strings cannot be squares,
        so they are all in $\overline{L}_{sq}$.
        To generate these, we add rules to generate strings of odd length,
        without restriction to the middle symbol (as defined in $R_X$).

        \step
        To handle strings of even length that are not squares, we use the
        property that they must be decomposable into two odd-sized strings
        with differing middle symbols.
        We add the rules $S \derives AB \mid BA$ to generate such strings,
        where $A$ generates odd-length strings with a $0$ as a middle symbol
        while $B$ generates odd-length strings with a $1$ as a middle symbol.
      \end{Answer}
  \end{enumalph}
\end{problem}
