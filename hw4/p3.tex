\begin{problem}
  Give an alternate proof, using CFGs alone (no PDAs),
  to prove that context-free grammars are closed under:
  \begin{enumalph}
    \item union.
    \begin{Answer}
      Let $G_1 = (V_1, \Sigma_1, R_1, S_1)$ and $G_2 = (V_2, \Sigma_2, R_2, S_2)$
      such that $G_1$ and $G_2$ generate $L_1$ and $L_2$, respectively.
      Define $G = (V, \Sigma, R, S)$ as follows:
      \begin{itemize}
        \item $V = V_1 \cup V_2 \cup \set{S}$, where $S \notin V_1 \cup V_2$ and $V_1 \cap V_2 = \emptyset$.
        \item $\Sigma = \Sigma_1 \cup \Sigma_2$
        \item $R = R_1 \cup R_2 \cup \set{(S, S_1), (S, S_2)}$
      \end{itemize}
      \begin{claim}
        $G$ generates $L_1 \cup L_2$.

        \begin{proof}
          We show that $G$ is a CFG that generates $L_1 \cup L_2$.
          \begin{enumroman}
            \item \textbf{Completeness:}
              Let $w$ be a string in $L_1 \cup L_2$.
              This means that, either:
              \begin{itemize}
                \item $w \in L_1$, so there exists some derivation $S_1 \derives^* w$ from $G_1$, or
                \item $w \in L_2$, so there exists some derivation $S_2 \derives^* w$ from $G_2$.
              \end{itemize}
              Note that $G$ is defined such that $V_1 \subset V$ and $V_2 \subset V$.
              Likewise, $R_1 \subset R$ and $R_2 \subset R$.
              Therefore, any such derivation can be deduced in $G$
              \emph{starting from the relevant symbol, of either $S_1$ or $S_2$}.
              However, the start symbol in $G$ is $S$, so a derivation
              $S \derives^* S_1$ or $S \derives^* S_2$ is needed to be able to
              derive strings from $L_1$ or $L_2$ respectively.
              Since the definition of $G$ adds two new rules, $(S, S_1)$ and $(S, S_2)$,
              the derivation $S \derives^* S_1$ and $S \derives^* S_2$ are possible.
              So any string that can be generated by $G_1$ can also be generated by $G$,
              and any string that can be generated by $G_2$ can also be generated by $G$,
              meaning $G$ can generate any string in $L_1 \cup L_2$.
            \item \textbf{Soundness:}
              If a string is generated by $G$, we claim that it is in $L_1 \cup L_2$.
              Note that $G$ has a single start symbol, $S$,
              and the only rules including $S$ are $(S, S_1)$ and $(S, S_2)$.
              This means from $S$ we can only derive
              \emph{either $S_1$ or $S_2$, but not both, and not any other symbol.}
              Since $S_1 \in V_1$ and $S_2 \in V_2$ and we defined $V_1$ and $V_2$
              to be disjoint, the only strings that can be generated from $S_1$
              must be in $L_1$ (using the rules in $R_1$)
              and the only strings that can be generated from $S_2$
              must be in $L_2$ (using the rules in $R_2$).
              Therefore, any string that can be generated by $G$ must be either
              in $L_1$ or in $L_2$, meaning any string $G$ generates is in $L_1 \cup L_2$.
          \end{enumroman}
        \end{proof}
      \end{claim}
    \end{Answer}
    \newpage
    \item concatenation.
    \begin{Answer}
      Let $G_1 = (V_1, \Sigma_1, R_1, S_1)$ and $G_2 = (V_2, \Sigma_2, R_2, S_2)$
      such that $G_1$ and $G_2$ generate $L_1$ and $L_2$, respectively.
      Define $G = (V, \Sigma, R, S)$ as follows:
      \begin{itemize}
        \item $V = V_1 \cup V_2 \cup \set{S}$, where $S \notin V_1 \cup V_2$ and $V_1 \cap V_2 = \emptyset$.
        \item $\Sigma = \Sigma_1 \cup \Sigma_2$
        \item $R = R_1 \cup R_2 \cup \set{(S, S_1S_2)}$
      \end{itemize}

      \begin{claim}
        $G$ generates $L_1L_2$.

        \begin{proof}
          We show that $G$ is a CFG that generates $L_1L_2$.
          \begin{enumroman}
            \item \textbf{Completeness:}
              Let $w$ be a string in $L_1L_2$.
              This means that, for some $u \in L_1$ and $v \in L_2$,
              $w = uv$.
              Therefore, there exists some derivation $S_1 \derives^* u$ in $G_1$
              and some derivation $S_2 \derives^* v$ from $G_2$.
              Since  $V_1 \subset V$, $V_2 \subset V$,
              $R_1 \subset R$, and $R_2 \subset R$,
              these derivations are also possible in $G$
              \emph{starting from the relevant symbol, of either $S_1$ or $S_2$}.
              But the start symbol in $G$ is $S$, so a derivation
              $S \derives^* S_1S_2$ is needed to be able to
              derive strings from $L_1L_2$.
              The definition of $G$ adds this rule, $(S, S_1S_2)$,
              so the derivation $S \derives^* S_1S_2$ is possible.
              Therefore, any string in $L_1L_2$ can be generated by $G$.
            \item \textbf{Soundness:}
              If a string is generated by $G$, we claim that it is in $L_1L_2$.
              $G$ has a single start symbol, $S$,
              and the only rule from $S$ is $(S, S_1S_2)$.
              This means from $S$ we can only derive $S_1S_2$.
              Since $S_1 \in V_1$ and $S_2 \in V_2$ and we defined $V_1$ and $V_2$
              to be disjoint, the only strings that can be generated from $S_1$
              must be in $L_1$ (using the rules in $R_1$)
              and the only strings that can be generated from $S_2$
              must be in $L_2$ (using the rules in $R_2$).
              Therefore, any string that can be generated by $G$ must be
              the concatenation of a string in $L_1$ and a string in $L_2$,
              so any string $G$ generates is in $L_1L_2$.
          \end{enumroman}
        \end{proof}
      \end{claim}
    \end{Answer}
    \newpage
    \item Kleene star.
    \begin{Answer}
      Let $G_1 = (V_1, \Sigma_1, R_1, S_1)$ be a CFG that generates $L$.
      Define $G = (V, \Sigma, R, S)$ as follows:
      \begin{itemize}
        \item $V = V_1 \cup \set{S}$, where $S \notin V_1$.
        \item $\Sigma = \Sigma_1$
        \item $R = R_1 \cup \set{(S, S_1S), (S, \eps)}$
      \end{itemize}

      \begin{claim}
        $G$ generates $L^*$.

        \begin{proof}
          We show that $G$ is a CFG that generates $L^*$.
          \begin{enumroman}
            \item \textbf{Completeness:}
              Let $w$ be a string in $L^*$.
              There are two possible scenarios:
              \begin{enumroman}
                \item $w = \eps$: Since we have the rule $S \derives \eps$,
                  $G$ can generate $\eps$.
                \item $w = w_1, \ldots, w_n$ with all $w_i \in L$.
                  This means that there exists some derivation $S_1 \derives^* w_1$ in $G_1$,
                  $S_1 \derives^* w_2$ in $G_1$, \ldots, and $S_1 \derives^* w_n$ in $G_1$.
                  Since $V_1 \subset V$ and $R_1 \subset R$,
                  each one of these derivations is also possible in $G$
                  \emph{starting from the relevant symbol, $S_1$}.
                  To derive their concatenations starting from $S$,
                  we need a rule that can recursively derive $S_1$
                  multiple ties from $S$.
                  We define this rule in the definition of $G$ as $(S, S_1S)$,
                  allowing $G$ to derive $S_1S_1S_1 \ldots S_1S$ from $S$,
                  then eventually replace the $S$ with $\eps$
                  and derive each $w_i$ from the corresponding $S_1$.
              \end{enumroman}
            \item \textbf{Soundness:}
              If a string is generated by $G$, we claim that it is in $L^*$.
              $G$ has a single start symbol, $S$,
              which yields either $\eps$ or $S_1S$.
              the first case generates $\eps$, which is in $L^*$.
              In the second case, repeated expansion of $S$ in the expression
              yields $S_1S_1S_1 \ldots S_1S$.
              Each $S_1$ eventually yields a string in $L$,
              and the final $S$ yields $\eps$.
              Therefore, any string that can be generated by $G$ must either be
              the empty string or a concatenation of strings from $L$ ---
              meaning it is in $L^*$.
          \end{enumroman}
        \end{proof}
      \end{claim}
    \end{Answer}
  \end{enumalph}
\end{problem}
