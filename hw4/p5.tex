\begin{problem}
  Let $\Sigma$ be an alphabet, $L \subseteq \Sigma^*$,
  and $\# \notin \Sigma$.
  Define the language
  \[ 
    \textsc{Intersperse}(\#, L) := 
    \set{a_1 \# a_2 \# \ldots \# a_n},
    \text{ each $a_i \in \Sigma$ and $a_1a_2 \ldots a_n \in L$}.
  \]

  \step
  Let $M_1 = (Q, \Sigma, \Gamma, \delta, q_0, F)$ be a PDA
  that recognizes $L$.
  Formally describe a PDA that recognizes $\textsc{Intersperse}(\#, L)$.
  Also give a high-level proof that your PDA works correctly.
\end{problem}
\begin{Answer}
  Let $M_2 = (\bigcup_{q \in Q} \set{q, q_\#, q_\eps}, \Sigma \cup \set{\#}, \Gamma, \delta_2, q_0, F_2)$ be a PDA
  such that:
  \begin{itemize}
    \item $F_2 = \set{q_\eps: q \in F} \cup \set{q_\#: q \in F}$
    \item $\delta_2$ is defined as follows:
    \begin{align*}
      \delta_2(q, \eps, \gamma) &= \delta(q, a, \gamma)_\eps \\
      \delta_2(q, x, \gamma) &= \delta(q, a, \gamma)_\# \text{ if $x \in \Sigma$} \\
      \delta_2(q_\#, \#, \eps) &= q \\
      \delta_2(q_\eps, \eps, \eps) &= q \\
    \end{align*}
  \end{itemize}
  \begin{figure}[H]
    \begin{tikzpicture}
      \node[state, initial] (a) at (0, 0) {$a$};
      \node[state] at (10, 0) (b) {$b$};
      \node[state] at (5, 3) (be) {$b_{\eps}$};
      \node[state] at (5, -3) (bh) {$b_{\#}$};
      \draw (a) edge[above, pos=0.5, dashed] node {$\begin{aligned}\eps, m \to n \\ x, y \to z \end{aligned}$} (b)
            (a) edge[below, pos=0.3] node {\crim{$x, y \to z$}} (bh)
            (a) edge[above, pos=0.3] node {\crim{$\eps, m \to n$}} (be)
            (bh) edge[below, pos=0.3] node {\green{$\#, \eps \to \eps$}} (b)
            (be) edge[above, pos=0.3] node {\green{$\eps, \eps \to \eps$}} (b);
    \end{tikzpicture}
    \caption{Modification of Transitions to Require $\#$ between any two non-epsilon Symbols.}
  \end{figure}

  \newpage
  \begin{claim}
    $M_2$ recognizes $\textsc{Intersperse}(\#, L)$.

    \step
    \begin{proof}
      In contrast to $M_1$, $M_2$ makes the following modifications:
      \begin{enumroman}
        \item Any transition that takes the PDA from a state $a$ to a state $b$,
          now does one of the following:
          \begin{itemize}
            \item Takes the PDA from $a$ to $b_\eps$ if the transition reads an epsilon input.
              From $b_\eps$, the PDA can choose to take a transition to $b$
              without reading any input or interfering with the stack, so $b$ is still reachable.
            \item Takes the PDA from $a$ to $b_\#$ if the transition reads a non-epsilon input.
              From $b_\#$, the only transition forward reads a $\#$ symbol but does not interfere
              with the stack, thus ensuring that $b$ is still reachable only if a $\#$ symbol
              exists between the consecutive letters in the input string.
          \end{itemize}
        \item For all accepting states $f \in F_1$,
          we define the states $f_\#$ and $f_\eps$ to be accepting states in $M_2$,
          and make $f$ non-accepting in $M_2$.
          This ensures that:
          \begin{itemize}
            \item If $a_1a_2 \ldots a_n$ is accepted in $M_1$, then $a_1 \# a_2 \# \ldots \# a_n$ is accepted in $M_2$.
            \item We do not accept strings ending with a $\#$ symbol, e.g. $a_1 \# a_2 \# \ldots \# a_n \#$, since
              the state $f$ itself is non-accepting.
          \end{itemize}
      \end{enumroman}
    \end{proof}
  \end{claim}
\end{Answer}
