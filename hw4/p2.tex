\begin{problem}
  In class, we wrote a formal construction of a PDA that proves
  that context-free languages are closed under union.
  Give similar constructions for PDAs to prove closure under:
  \begin{enumalph}
    \item concatenation.
      \begin{Answer}
        Let $L_1$ and $L_2$ be context-free languages.
        Take $M_1 = (Q_1, \Sigma_1, \Gamma_1, \delta_1, q_{01}, F_1)$
        to be a PDA that recognizes $L_1$ and 
        $M_2 = (Q_2, \Sigma_2, \Gamma_2, \delta_2, q_{02}, F_2)$
        to be a PDA that recognizes $L_2$.

        \step
        Construct a new PDA $M = (Q, \Sigma, \Gamma, \delta, q_0, F)$ as follows:
        \begin{itemize}
          \item $Q = Q_1 \cup Q_2$ \quad (after enforcing that $Q_1 \cap Q_2 = \emptyset$)
          \item $\Sigma = \Sigma_1 \cup \Sigma_2$ (we don't particularly care about equality of $\Sigma_1$ and $\Sigma_2$)
          \item $\Gamma = \Gamma_1 \cup \Gamma_2$ (we don't particularly care about equality of $\Gamma_1$ and $\Gamma_2$)
          \item $q_{0} = q_{01}$
          \item $F = F_2$
          \item $\delta$ is defined as follows:
            \begin{align*}
              \delta(q, a, \gamma) = 
              \begin{cases}
                \delta_1(q, a, \gamma) \cup \set{q_{02}, \emptyset} & \text{ if $q \in F_1$ and $a = \gamma = \eps$.} \\
                \delta_1(q, a, \gamma) & \text{ if $q \in Q_1$. } \\
                \delta_2(q, a, \gamma) & \text{ if $q \in Q_2$. }
              \end{cases} 
            \end{align*}
        \end{itemize}
        \begin{claim}
          $M$ recognizes $L_1 \cup L_2$.
          
          \begin{proof}
            Note that the starting state of $M$ is $q_{01}$,
            while the accepting states of $M$ are in $F_2$.
            The only transition that takes $M$ from a state formerly in $Q_1$
            to a state formerly in $Q_2$ is when (1) we are at a state $q \in F_1$
            (an accepting state of $M_1$), (2) we read no input (epsilon transition),
            and (3) we clear the stack.

            \step
            Since $M_1$ recognizes $L_1$, we know that the computational path of
            $M_1$ on all strings in $L_1$ ends in an accepting state $q_{f1} \in F_1$.
            Likewise, since $M_2$ recognizes $L_2$, we know that the computational path of
            $M_2$ on all strings in $L_2$ ends in an accepting state $q_{f2} \in F_2$.
            
            \begin{itemize}
              \item \textbf{Completeness:} If a string $s$ is in $L_1L_2$,
                then we can write it as $s = xy$ for some $x \in L_1$ and $y \in L_2$.
                Then, the computational path of $M$ on $x$ mimics that of $M_1$
                (since it starts at $q_{01}$ and we use $\delta_1$ for all states $q \in Q_1$).
                Therefore, $M$ has a computational path on $x$ that ends in an accepting state $q_{f1} \in F_1$.
                Then, $M$ takes the epsilon transition to $q_{02}$.
                In processing $y$, $M$ starts at $q_{02}$ and uses $\delta_2$ for all states $q \in Q_2$,
                so it has some computational path from $q_{02}$ to $q_{f2} \in F_2$.
                Putting these two paths together and the middle epsilon transition,
                we get a computational path of $M$ from $q_{01}$ to $q_{f2} \in F_2$,
                which is an accepting state of $M$.
                Therefore, $M$ accepts the string.
              \item \textbf{Soundness:} Let $s$ be a string accepted by $M$.
                Then there must exist some computational path $p_1$ of $M$ on $s$, taking $M$ from
                $q_{01}$ to a state in $F_1$, followed by the epsilon transition to $q_{02}$,
                and some computational path $p_2$ of $M$ from $q_{02}$ to a state in $F_2$.
                By definition of $M$, $p_1$ corresponds to an accepting
                computational path of $M_1$ and $p_2$ corresponds to an accepting
                computational path of $M_2$,
                Meaning that $M_1$ accepts some prefix $x$ of $s$ and $M_2$ accepts some suffix $y$ of $s$,
                and $x$ and $y$ form the entire string $s$, so $s = xy, x \in L_1, y \in L_2$.
                Therefore, any such $s$ accepted by $M$ is in $L_1L_2$.
            \end{itemize}
          \end{proof}
        \end{claim}
      \end{Answer}
    \item Kleene star.
      \begin{Answer}
        Let $L$ be a context-free language.
        Take $M = (Q, \Sigma, \Gamma, \delta, q_0, F)$ to be a PDA that recognizes $L$.

        \step
        Construct a new PDA $M_2 = (Q \cup q_{start}, \Sigma, \Gamma, \delta_2, q_0, F \cup q_{start})$ where:
        \begin{itemize}
          \item $q_{start} \notin Q$
          \item $\delta_2$ is defined as follows:
            \begin{align*}
              \delta_2(q, a, \gamma) = 
              \begin{cases}
                \delta(q, a, \gamma) \cup \set{(q_0, \emptyset)} & \text{ if $q \in F$ and $a = \gamma = \eps$. } \\
                \set{(q_0, \emptyset)} & \text{ if $q = q_{start}$ and $a = \gamma = \eps$. } \\
                \delta(q, a, \gamma) & \text{ otherwise. } \\
              \end{cases} 
            \end{align*}
          \end{itemize}
          \begin{claim}
            $M_2$ recognizes $L^*$.
            \begin{proof}
              Note that:

              \step
              \textbf{Completeness:}
                If a string $s$ is in $L^*$, then, either:
                \begin{enumroman}
                  \item $s = \eps$. Since $q_{start}$ is an accepting state of $M_2$,
                    $M_2$ accepts $s$.
                  \item $s = x_1x_2 \cdots x_n$, with all $x_i \in L$.
                    Then, for each $x_i$, there exists some computational path that takes $M_2$
                    from $(q_0, \emptyset)$ to some $(q_f, \Gamma_0)$ where $q_f \in F$
                    and $\Gamma_0 \in \Gamma^*$.
                    Since $(q_0, \emptyset)$ is in the set of possible next states for epsilon
                    transitions on all accepting states (as defined in case $1$ of $\delta_2$),
                    we have a connecting path from any such $(q_f, \Gamma_0)$ to $(q_0, \emptyset)$
                    between any $x_i$ and $x_{i+1}$.
                    Therefore, $M_2$ has some computational path that:
                    \begin{enumroman}
                      \item starts at $(q_{start}, \emptyset)$,
                      \item Advances to some accepting state $(q_f, \Gamma_0)$ after reading $x_1$,
                      \item Takes an $\eps$-transition back to to $(q_0, \emptyset)$,
                      \item Advances to some accepting state $(q_{f2}, \Gamma_1)$ after reading $x_2$,
                      \item Takes an $\eps$-transition back to to $(q_0, \emptyset)$ again,
                      \item Repeats the process for $x_3, \ldots, x_n$, and
                      \item Is in some accepting state after finishing reading $x_n$ (but not taking
                        the$\eps$-transition back to $(q_0, \emptyset)$).
                    \end{enumroman}
                    Therefore, $M_2$ has a computational that accepts $s$, so $M_2$ accepts $s$.
                \end{enumroman}
              
              \step
              \textbf{Soundness:}
                If $s$ is accepted by $M_2$, then, either:
                \begin{enumroman}
                  \item $s = \eps$, since $q_{start}$ is an accepting state.
                    Since $\eps \in L^*$ for any language, then $s$ is a valid
                    string in $L^*$.
                  \item Otherwise, we claim that $s = x_1x_2 \cdots x_n$,
                    with all $x_i \in L$.

                    $M_2$ mimics the transitions of $M$, only adding a new start state
                    and $\eps$-transitions from accepting states to the old start state.
                    Therefore, if $s$ is accepted by $M_2$, then a suffix $x_1$ of $s$
                    (which might be the whole string) takes $M_2$ from $(q_0, \emptyset)$
                    to some accepting state $(q_f, \Gamma_0)$ where $q_f \in F$.
                    If the suffix is NOT the whole string, write the whole string $s$ as
                    $s = px_1$, then if we erase the suffix $x_1$ then the prefix $p$
                    must also end up in some accepting state of $M_2$.
                    Continuing in the same way, we can extract a suffix $x_2$ of $p$,
                    and so on up to some $x_n$ where we remain with the empty string.
                    Therefore, we can write $s = x_nx_{n-1} \cdots x_1$ where all $x_i \in L$.
                    But this is exactly identical to writing $s = x_1x_2 \cdots x_n$
                    required for $s$ to be in $L^*$,
                    only that the numbering of $x$ is reversed.
                    Therefore, any such accepted string $s$ must be in $L^*$.
                \end{enumroman}
            \end{proof}
          \end{claim}
      \end{Answer}
  \end{enumalph}
\end{problem}
