\begin{problem}
  Consider the following CFG:
  \[
    S \to 1S00 \mid 00S1 \mid SS \mid 0S1S0 \mid \eps
  \]
  \begin{enumalph}
    \item Give a simple description of the language it generates
      using set-builder notation.
      \begin{Answer}
        \[
          L = \set{x \in \set{0, 1}^* : N_0(x) = 2N_1(x)}
        \]
      \end{Answer}
    \item Now for the hard and fun part: prove the correctness
      of your answer.
      \begin{Answer}
        We shall prove this in two parts. First, we show by induction on the form
        of the string that the condition holds for all the derivations of $S$,
        then we show that every string can be written as one of the derivations of $S$.

        \step
        \begin{claim}
          \textbf{$N_0(S) = 2N_1(S)$ for all derivations of $S$.}
          \begin{proof}
            We shall show this by induction on the form of $S$.

            \step
            \textbf{Base Case:} $S \derives \eps$
              
            \step
            In this case, $N_0(S) = 0$ and $N_1(S) = 0$,
            so the condition $N_1(S) = 2N_0(S)$ holds.

            \step
            \textbf{Inductive Step:}
            \begin{enumroman}
              \item $S \derives 1\crim{S}00$

                \step
                Assuming $N_0(\crim{S}) = 2N_1(\crim{S})$ for the smaller case,
                let $x = N_1(\crim{S})$.

                \step
                Then $N_0(S) = N_0(\crim{S}) + 2 = 2x + 2 = 2(x + 1)$,
                and $N_1(S) = N_1(\crim{S}) + 1 = x + 1$.

                \step
                Therefore, $N_0(S) = 2N_1(S)$ and the condition holds.
              \item $S \derives 00\crim{S}1$
              
                \step
                Assuming $N_0(\crim{S}) = 2N_1(\crim{S})$ for the smaller case,
                proceeding as in case (i) above we also see that $N_0(S) = 2N_1(S)$.
              \item $S \derives \crim{S}\green{S}$
                
                \step
                Assuming that $N_0(\crim{S}) = 2N_1(\crim{S})$ and
                $N_0(\green{S}) = 2N_1(\green{S})$ for the two smaller cases,\\
                let $x = N_1(\crim{S})$ and $y = N_1(\green{S})$.

                \step
                We see that $N_0(S) = N_0(\crim{S}) + N_0(\green{S}) = 2x + 2y = 2(x + y)$,
                and $N_1(S) = N_1(\crim{S}) + N_1(\green{S}) = x + y$.

                \step
                Therefore, $N_0(S) = 2N_1(S)$ and the condition holds.
              \item $S \derives 0\crim{S}1\green{S}0$
              
                \step
                Assuming that $N_0(\crim{S}) = 2N_1(\crim{S})$ and
                $N_0(\green{S}) = 2N_1(\green{S})$ for the two smaller cases,\\
                let $x = N_1(\crim{S})$ and $y = N_1(\green{S})$.

                \step
                Then $N_0(S) = N_0(\crim{S}) + N_0(\green{S}) + 2 = 2x + 2y + 2 = 2(x + y + 1)$,
                and $N_1(S) = N_1(\crim{S}) + N_1(\green{S}) + 1 = x + y + 1$.

                \step
                Therefore, $N_0(S) = 2N_1(S)$ and the condition holds.
            \end{enumroman}
          \end{proof}
        \end{claim}
        \begin{claim}
          Every string $x \in L$ can be expressed as a derivation of $S$.

          \begin{proof}

            \step
            First, let's define a useful metric.
            For any string $x$, let $f(x) = N_0(x) - 2N_1(x)$.
            This means that $f(x) = 0$ if and only if $x \in L$.

            Now let's look at the strings in $L$ and see how the metric changes
            over the length of the strings.
            Let $x$ be an arbitrary string in $L$.
            There are two general cases:
            \begin{enumroman}
              \item $x$ starts and ends with the same symbol.
                Write $x = x_1x_2 \ldots x_n$ so that $x_1 = x_n$.
                Since $x \in L$, $f(x) = 0$.
                \begin{itemize}
                  \item If $x_1 = 0$ and $x_n = 0$, that means
                    $f(x_1) = 1$ and $f(x_1x_2\ldots x_{n-1}) = -1$.
                    For the metric to move from $1$ to $-1$, we must have
                    crossed the zero line at some point. However, since the function
                    decreases by $2$, there are 2 possible scenarios:
                    \begin{itemize}
                      \item If the metric equals $0$ at some point,
                        we can split the string as in the previous case.
                      \item If the metric does not equal $0$ at any point,
                        then it must transition from $1$ to $-1$ at the point
                        where it crossed the zero line (say, $k$).
                        Therefore, ignoring the $0$ causing the decrease
                        and the zeros at the beginning and the end of the string,
                        we have that $f(x_2\ldots x_{k-1}) = 0$
                        and $f(x_{k+1}\ldots x_{n-1}) = 0$.
                        Therefore, we can write the string as $s = 0S1S0$.
                    \end{itemize}
                  \item If $x_1 = 1$ and $x_n = 1$, that means
                    $f(x_1) = -2$ and $f(x_1x_2\ldots x_{n-1}) = 2$.
                    For the metric to move from $-2$ to $2$, we must have
                    crossed the zero line. However, the metric always
                    increases by $1$ so there must exist some $j$ between $1$ and $n-1$
                    so that $f(x_1x_2\ldots x_j) = 0$.
                    We can therefore split the string $x_1x_2\ldots x_n$ into
                    $x_1x_2\ldots x_j$ and $x_jx_{j+1}\ldots x_n$, with each of the substrings
                    being a member of $L$.
                  
                \end{itemize}
                
              \item $x$ starts and ends with differing symbols.

                \step
                Write $x = x_1x_2 \ldots x_n$ so that $x_1 \neq x_n$.
                Suppose $x_0 = 0$ and $x_n = 1$.
                Then $f(x_1) = 1$ and $f(x_{n-1}) = 2$.
                if $x_2 = 1$, we can split the sentence into two!
                Therefore, $x_2$ must be $0$ so we can write the sentence as $00S1$.
                The same argument applies for when $0$ and $1$ are interchanged.
                % Then $f(x_1) = -1$ and $f(x_n) = -2$ (or vice versa if the symbols are interchanged),
                % implying the string \emph{must} pass through the zero line.
                % If it passes through the origin, then we can split it into two sentences
                % and apply the previous case.
                % If it does not pass through the origin, then 
                % In this case, the string must either have a double occurrence of $0$
                % at the beginning or at the end, so we can write it as either
                % $S = 00S1$ or $S = 1S00$.
            \end{enumroman}
          \end{proof}
        \end{claim}
      \end{Answer}
  \end{enumalph}
\end{problem}
