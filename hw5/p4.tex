\begin{problem}
  Formally describe a two-tape NDTM for the language
  $\set{ ww : w \in \set{0, 1}^*}$.
  Again, provide a diagram, plus explanations and comments.

  \step
  Appreciate the ease of programming resulting from nondeterminism
  and the availability of two tapes.
\end{problem}
\begin{Answer}
  \textbf{Strategy:}
  Let tape $A$ and tape $B$ be the two tapes in the TM.
  Note that at the start, the input string is on tape $A$
  (plus infinite blanks on the right end), and tape $B$ is blank.

  \begin{enumroman}
    \item Add a left-end marker ($\#$) to tape $B$ and move the head
      on tape $B$ one step right.

    \item Traverse tape $A$ from left to right, copying the contents of tape $A$
      onto tape $B$.

    \item Non-deterministically, stop copying the contents of tape $A$ onto tape $B$,
      move the head on tape $B$ to the left-most end (but maintain the head on tape $A$
      at its current position) and compare the remaining contents on tape $A$
      with those copied onto tape $B$.

    \item Accept if there is such a computational path that copies some (possibly empty)
      part of the input string onto tape $B$, compares it with the remaining part
      of the string on tape $A$, and successfully reaches the end of the string.
  \end{enumroman}

  \step
  \emph{
    Note: In the turing machine diagram,
    when we denote $(x, y)$ to as readings from the two tapes
    such that $x, y \in \set{0, 1}$ and $x \neq y$
    (otherwise, we denote the reading as $(x, x)$ to specify equality).
    We denote $\none$ when the reading from the tape is blank.
  }
    
  
  \newpage
  \bigskip
  \step
  Note: similar to NFAs, I allowed the NDTM to crash on unexpected inputs
  rather than explicitly handling them / rejecting.
    
  \begin{figure}[H]
    \centering
    \begin{tikzpicture}
      \node[state, initial] (q0) at (0, 0) {$q_{start}$};
      \node[state] (q1) at (6, 0) {$q_{copy}$};
      \node[state] (q2) at (6, -6) {$q_{rewind-B}$};
      \node[state] (q3) at (6, -12) {$q_{compare}$};
      \node[state, accepting] (q4) at (0, -12) {$q_{accept}$};
      % \node[state] (q5) at (12, -12) {$q_{reject}$};

      \draw (q0) edge[above] node {$(x, \none) \to (x, \#), (S, R)$} (q1)
            (q1) edge[loop right, right] node {$(x, \none) \to (x, x), (R, R)$} (q1)
            (q1) edge[left] node {$(x, \none) \to (x, \none), (S, S)$} (q2)
            (q2) edge[loop right, right] node {$
              \begin{aligned}
                (x, x) \to (x, x), (S, L)\\
                (x, y) \to (x, y), (S, L)
              \end{aligned}
            $} (q2)
            (q2) edge[left, pos=0.3] node {$(x, \#) \to (x, \#), (S, R)$} (q3)
            (q3) edge[loop below, below] node {$(x, x) \to (x, x), (R, R)$} (q3)
            % (q3) edge[above] node {$
            %   \begin{aligned}
            %     (x, y) \to (x, y), (S, S) \\
            %     (x, \none) \to (x, \none), (S, S) \\
            %     (\none, x) \to (\none, x), (S, S)
            %   \end{aligned}    
            % $} (q5)
            (q3) edge[above] node {$(\none, \none) \to (\none, \none), (S, S)$} (q4)
            (q0) edge[right] node {$(\none, \none) \to (\none, \none), (S, S)$} (q4);
    \end{tikzpicture}
    \caption{Turing Machine to decide $L = \set{ww : w \in \set{0, 1}^*}$}
    \label{fig:p4}
  \end{figure}
\end{Answer}
