\begin{problem}
  Show that a k-tape TM $M$ can be simulated by a single-tape TM $M'$
  in such a way that a computation that takes time $t$
  (i.e., $t$ steps of one configuration yielding another)
  on $M$ takes time $O(t^2)$ on $M'$.
  The big-O notation may hide a constant that depends on $k$.
  You may assume that $t \geq \abs{x}$, where $x$ is the input to $M$.
\end{problem}

\begin{Answer}
  Let $M$ be a k-tape TM over some alphabet $\Sigma$.

  \step
  A single-tape TM $M'$ can simulate $M$ as follows:
  \begin{enumroman}
    \item $M'$ copies the contains of all the tapes of $M$ onto its single
      tape, with a separator between the contents of each tape.
      For instance, if the contents of the $k$ tapes of $M$
      are \begin{align*}
        t_1 &= \sigma_{(1, 1)}, \sigma_{(1, 2)} \ldots \sigma_{(1, n_1)} \\
        t_2 &= \sigma_{(2, 1)}, \sigma_{(2, 2)} \ldots \sigma_{(2, n_2)} \\
        \vdots \\
        t_k &= \sigma_{(k, 1)}, \sigma_{(k, 2)}, \ldots, \sigma_{(k, n_k)}
      \end{align*}
      with each $\sigma_{(i, j)} \in \Sigma$,
      then the corresponding configuration of $M'$ is
      \[ 
        \#, \sigma_{(1, 1)}, \sigma_{(1, 2)} \ldots \sigma_{(1, n)},
        \#, \sigma_{(2, 1)}, \sigma_{(2, 2)} \ldots \sigma_{(2, n)},
        \#, \ldots, 
        \#, \sigma_{(k, 1)}, \sigma_{(k, 2)}, \ldots, \sigma_{(k, n_k)} \#
      \]
      where $\#$ is a special symbol that is not in the alphabet of $M$.
    \item $M'$ tracks multiple head positions (within each sub-tape)
      for each of the $k$ tapes of $M$.
      For instance, in the above example, if the heads of $M$ are at
      the beginning of each tape, then the configuration n $M'$
      (using $H_i$ to denote the head position on tape $i$)
      would be:
      \[
        \#, H_1, \sigma_{(1, 1)}, \sigma_{(1, 2)} \ldots \sigma_{(1, n)},
        \#, H_2, \sigma_{(2, 1)}, \sigma_{(2, 2)} \ldots \sigma_{(2, n)},
        \#, \ldots, 
        \#, H_k, \sigma_{(k, 1)}, \sigma_{(k, 2)}, \ldots, \sigma_{(k, n_k)} \#
      \]

    \item On each transition, $M$ reads some $k$-tuple of inputs from the $k$ tapes.
      transitions to some new state $q$, and write some $k$-tuple of updates
      to the $k$ tapes.
      in turn, $M'$ simulates a sequence of transitions that updates
      each of the $k$ positions representing the head $k$ head positions
      on its single tape.
  \end{enumroman}

  \newpage
  \step
  For a computation that runs in $t$ steps on $M$;
  \begin{itemize}
    \item $M'$ copies the contents of the $k$ tapes of $M$
      onto its single tape and sets the head positions to the beginning
      of each sub-tape in $k\cdot\abs{x}$ steps.
    \item For each of the subsequent $k-1$ steps of $M$, $M'$
      simulates a computation that updates each sub-tape.
      In the worst case, this requires traversing traversing
      the entire tape of $M'$, which which takes $k\cdot\abs{x}$ steps.
      This is repeated $t-1$ times for a total of
      $(t-1)\cdot k\cdot\abs{x}$ steps.
    \item In total, $M'$ takes $k\cdot\abs{x} + (t-1)\cdot (k\cdot\abs{x})
      = t \cdot k\cdot\abs{x}$
      steps to simulate the computation of $M$.\\
      Assuming $t \geq \abs{x}$, the simulation runs in time $t' \leq kt^2$ steps,
      which is $O(t^2)$. 
      
  \end{itemize}


\end{Answer}
