\begin{problem}
Draw a state diagram of a DFA that recognizes the language consisting of all strings
in $\set{0, 1}^*$ such that each string is of length at least three
and every block of three consecutive symbols has at least one $0$.

\noindent
\emph{(Thus, for example, $0011001$ is in the language, but $0011100$ is not.)}

\step
Explain, in one or two sentences, the idea behind your DFA construction
\end{problem}

\begin{Answer}
  To make this problem easier to reason about,
  I split it into three incremental steps:
  \begin{enumroman}
    \item Construct a DFA that accepts all strings of length at least $3$.
    \item Construct a DFA that accepts all strings with at least one $0$ in each block of $3$ letters.
    \item Combine the two DFAs to get the final DFA.
  \end{enumroman}

  \step
  \textbf{Part 1}: DFA to accept all strings of length at least $3$.

  \begin{figure}[H]
    \centering
    \begin{tikzpicture}
      \node[state, initial] (q0) at (0, 0) {$q_0$};
      \node[state, right of=q0] (q1) {$q_1$};
      \node[state, right of=q1] (q2) {$q_2$};
      \node[state, accepting, right of=q2] (q3) {$q_3$};
      \draw (q0) edge[above] node {$0, 1$} (q1)
            (q1) edge[above] node {$0, 1$} (q2)
            (q2) edge[above] node {$0, 1$} (q3)
            (q3) edge[loop right, right] node {$0, 1$} (q3);
    \end{tikzpicture}
    \caption{Accept all strings of length at least $3$.}
    \label{fig:at-least-3}
  \end{figure}

  \newpage
  \step
  \textbf{Part 2}

  \step DFA to accept all strings with at least one $0$ in each block of $3$ letters.

  \step
  The simplest idea I had was to trap and reject once we encounter
   $111$ since that is the only non-accepting $3$-sequence block.

  \begin{figure}[H]
    \centering
    \begin{tikzpicture}
      \node[state, accepting, initial] (q0) at (0, 0) {$q_0$};
      \node[state, accepting, right of=q0] (q1) {$q_1$};
      \node[state, accepting, right of=q1] (q2) {$q_2$};
      \node[state, right of=q2] (q3) {$q_3$};
      \draw (q0) edge[loop above, above] node {$0$} (q0)
            (q0) edge[above] node {$1$} (q1)
            (q1) edge[bend left, below] node {$0$} (q0)
            (q1) edge[above] node {$1$} (q2)
            (q2) edge[bend right, above] node {$0$} (q0)
            (q2) edge[above] node {$1$} (q3)
            (q3) edge[loop right, right] node {$0, 1$} (q3);
    \end{tikzpicture}
    \caption{Accept all strings with at least one $0$ in each block of $3$.}
    \label{fig:at-least-one-0-in-each-block-of-3}
  \end{figure}

  % \newpage
  \step
  While working on $Q_6$, I realized that the
  multiple-zeros scenario necessitates tracking the full state
  since we will not necessarily have a single reject state.
  So I redesigned this part of the DFA to track the full state.
  
  \begin{blockcolor}
  \begin{figure}[H]
    \centering
    \begin{tikzpicture}
      \node[state, accepting, initial] (q000) at (0, 0) {$q_{000}$};
      \node[state, accepting, right of=q000] (q001) {$q_{001}$};
      \node[state, accepting, right of=q001] (q010) {$q_{010}$};
      \node[state, accepting, right of=q010] (q011) {$q_{011}$};
      \node[state, accepting, below of=q001] (q100) {$q_{100}$};
      \node[state, accepting, below of=q010] (q101) {$q_{101}$};
      \node[state, accepting, below of=q011] (q110) {$q_{110}$};
      \node[state, right of=q110] (q111) {$q_{111}$};
      \draw (q000) edge[loop above, above] node {$0$} (q000)
            (q000) edge[above] node {$1$} (q001)
            (q001) edge[above] node {$0$} (q010)
            (q001) edge[bend left, above] node {$1$} (q011)
            (q010) edge[above] node {$0$} (q100)
            (q010) edge[bend left, right, pos=0.25] node {$1$} (q101)
            (q011) edge[right] node {$0$} (q110)
            (q011) edge[above] node {$1$} (q111)
            (q100) edge[left] node {$0$} (q000)
            (q100) edge[left] node {$1$} (q001)
            (q101) edge[bend left, left, pos=0.25] node {$0$} (q010)
            (q101) edge[above] node {$1$} (q011)
            (q110) edge[bend left, below] node {$0$} (q100)
            (q110) edge[above] node {$1$} (q101)
            (q111) edge[loop right, right] node {$0, 1$} (q111);
    \end{tikzpicture}
    \caption{Accept all strings with at least one $0$ in each block of $3$.}
    \label{fig:at-least-one-0-in-each-block-of-3-full}
  \end{figure}
  \end{blockcolor}

  % \newpage
  \step
  \textbf{Part 3}: Combine DFA ~\ref{fig:at-least-3}
  and DFA ~\ref{fig:at-least-one-0-in-each-block-of-3-full}.

  \step
  I debated whether to use the simplified DFA (figure \ref{fig:at-least-one-0-in-each-block-of-3})
  or the full-state DFA (figure \ref{fig:at-least-one-0-in-each-block-of-3-full}),
  eventually going with the full-state DFA as that would help
  when working on $Q_6$.

  \begin{figure}[H]
    \centering
    \resizebox{450pt}{!}{%
      \begin{tikzpicture}
        \node[state, initial] (q0-000) at (0, 0) {$q_{(0,\ \eps)}$};
        \node[state, below of=q0-000] (q1-000) {$q_{(1,\ 0)}$};
        \node[state, below of=q1-000] (q2-000) {$q_{(2,\ 00)}$};
        \crim{\node[state, accepting, below of=q2-000] (q3-000) {$q_{(3,\ 000)}$};}

        \crim{\node[state, accepting, right of=q3-000] (q3-001) {$q_{(3,\ 001)}$};}
        \crim{\node[state, accepting, right of=q3-001] (q3-010) {$q_{(3,\ 010)}$};}
        \crim{\node[state, accepting, right of=q3-010] (q3-011) {$q_{(3,\ 011)}$};}
        \crim{\node[state, accepting, right of=q3-011] (q3-100) {$q_{(3,\ 100)}$};}
        \crim{\node[state, accepting, right of=q3-100] (q3-101) {$q_{(3,\ 101)}$};}
        \crim{\node[state, accepting, right of=q3-101] (q3-110) {$q_{(3,\ 110)}$};}
        \crim{\node[state,            right of=q3-110] (q3-111) {$q_{(3,\ 111)}$};}
        
        \node[state, above of=q3-010] (q2-001) {$q_{(2,\ 01)}$};
        \node[state, above of=q3-100] (q2-010) {$q_{(2,\ 10)}$};
        \node[state, above of=q2-010] (q1-001) {$q_{(1,\ 1)}$};
        \node[state, above of=q3-110] (q2-011) {$q_{(2,\ 11)}$};
        
        \draw (q0-000) edge[left] node {$0$} (q1-000)
              (q0-000) edge[above] node {$1$} (q1-001)

              (q1-000) edge[left] node {$0$} (q2-000)
              (q1-000) edge[above] node {$1$} (q2-001)

              (q2-000) edge[left] node {$0$} (q3-000)
              (q2-000) edge[above] node {$1$} (q3-001)

              (q2-001) edge[left] node {$0$} (q3-010)
              (q2-001) edge[above] node {$1$} (q3-011);

        \draw (q1-001) edge[left] node {$0$} (q2-010)
              (q1-001) edge[above] node {$1$} (q2-011)

              (q2-010) edge[left] node {$0$} (q3-100)
              (q2-010) edge[above] node {$1$} (q3-101)

              (q2-011) edge[left] node {$0$} (q3-110)
              (q2-011) edge[above] node {$1$} (q3-111);
        \crim{
        \draw (q3-000) edge[loop left, left] node {$0$} (q3-000)
              (q3-000) edge[above] node {$1$} (q3-001)
              (q3-001) edge[below] node {$0$} (q3-010)
              (q3-001) edge[bend left, above, pos=0.25] node {$1$} (q3-011)
              (q3-010) edge[bend left, below, pos=0.90] node {$0$} (q3-100)
              (q3-010) edge[bend left, above, pos=0.85] node {$1$} (q3-101)
              (q3-011) edge[bend left, above, pos=0.95] node {$0$} (q3-110)
              (q3-011) edge[bend left, below, pos=0.95] node {$1$} (q3-111)
              (q3-100) edge[bend left, below, pos=0.85] node {$0$} (q3-000)
              (q3-100) edge[bend left, below, pos=0.85] node {$1$} (q3-001)
              (q3-101) edge[bend left, below, pos=0.95] node {$0$} (q3-010)
              (q3-101) edge[bend left, below, pos=0.75] node {$1$} (q3-011)
              (q3-110) edge[bend left, below] node {$0$} (q3-100)
              (q3-110) edge[above] node {$1$} (q3-101)
              (q3-111) edge[loop right, right] node {$0, 1$} (q3-111);
        }
      \end{tikzpicture}
    }
    \caption{Accept all strings with length at least $3$ and at least one $0$ in each block of $3$.}
    \label{fig:at-least-3-and-at-least-one-zero}
  \end{figure}

  \step
  \textbf{DFA Specification}
  \step
  For simplicity of expression, we define the following functions:
  \begin{align*}
    \shift{s} &= \begin{cases}
      \eps &\text{ if } s = \eps\\
      x &\text{ otherwise, where } s \text{ is parsed as } \crim{\set{0, 1}^1x} \\
    \end{cases} \\
    \beta(s) &= n \text{ where $n$ is the integer obtained by interpreting $s$ as a binary number.} \\
    \#(s) &= n \text{ where $n$ is the length of $s$.} \\
    \#_a(s) &= n \text{ where $n$ is the number of occurrences of $a$ in $s$.}
  \end{align*}
  \begin{align*}
    M &= \left( Q,\ \Sigma,\ \delta,\ q_0,\ F \right) \text{ where: }\\
    Q &= \set{ q_{(i,\ s)} : 0 \leq i \le 3,\; s \in \set{0, 1}^* \text{ and } \#(s) = i} \\
    \Sigma &= \set{0, 1} \\
    \delta(q_{(i,\ s)}, x) &= \begin{cases}
      q_{(i+1,\ sx)} &\text{ if } i < 3 \quad \quad \quad \quad \quad \quad \quad \zaff{\text{ // Still too short.}}\\
      q_{(i, \shift{s}x)} &\text{ if $i = 3$ and } \#_0(s) \ge 1 \quad \zaff{\text{ // Valid length and condition.}}\\
      q_{(i,\ s)} &\text{ otherwise.} \quad \quad \quad \quad \quad \quad \zaff{\text{// Invalid condition, trap in the state.}}
    \end{cases} \\
    q_0 &= q_{(0,\ \eps)} \\
    F &= \set{ q_{(3,\ s)} : \#_0(s) \ge 1}
  \end{align*}
\end{Answer}
