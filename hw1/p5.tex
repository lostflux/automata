\begin{problem}
  For a string $x \in {0, 1}^*$, let $\beta(x)$ denote the integer obtained
  by interpreting $x$ as a binary number.
  Thus, for example, $\beta(11001) = 25$ and $\beta(0011) = 3$.
  We also define $\beta(\epsilon) = 0$ for convenience.
  
  \noindent
  Design a DFA for the language $L = \set{ x \in \set{0, 1}^* : \beta(x) \text{ is divisible by } 5}$.
  \emph{Draw the state diagram and also specify the DFA formally}.
  \crim{Hint: For integers $m, n, p$, we have $(mn + p) \mod 5 = ((m \mod 5) \cdot n + p) \mod 5$}
\end{problem}
\begin{Answer}
  \begin{figure}[H]
    \centering
    \begin{tikzpicture}
      \node[state, initial, accepting] (q0) at (0, 0) {$q_0$};
      \node[state, right of=q0] (q2) {$q_2$};
      \node[state, below of=q0] (q1) {$q_1$};
      \node[state, right of=q2] (q4) {$q_4$};
      \node[state, below of=q4] (q3) {$q_3$};
      \draw (q0) edge[loop above, above] node {$0$} (q0)
            (q0) edge[left] node {$1$} (q1)
            (q1) edge[left] node {$0$} (q2)
            (q1) edge[bend left, above] node {$1$} (q3)
            (q2) edge[above] node {$0$} (q4)
            (q2) edge[above] node {$1$} (q0)
            (q3) edge[bend left, below] node {$0$} (q1)
            (q3) edge[above] node {$1$} (q2)
            (q4) edge[right] node {$0$} (q3)
            (q4) edge[loop above, above] node {$1$} (q4);
    \end{tikzpicture}
    \caption{Accept all strings $x$ where $\beta(x) \equiv 0 \pmod 5$.}
    \label{fig:divisible-by-5}
  \end{figure}
  \textbf{DFA Specification}\\
  For convenience, we define the function
  \[ \beta(x) = n \text{ where $n$ is the value of $x$ interpreted as a binary number.}\]
  
  \step
  We specify the DFA as follows:

  \begin{align*}
    M &= \left( Q,\ \Sigma,\ \delta,\ q_0,\ F \right) \text{ where: }\\
    Q &= \set{q_i : 0 \le i < 5}\\
    \Sigma &= \set{0, 1}\\
    \delta(q_i, x) &= q_j \text{ where } j = ((2i + \beta(x)) \mod 5)
    &= \begin{cases}
      q_0 & \text{if } i = 0 \text{ and } x = 0\\
      q_1 & \text{if } i = 0 \text{ and } x = 1\\
      q_2 & \text{if } i = 1 \text{ and } x = 0\\
      q_3 & \text{if } i = 1 \text{ and } x = 1\\
      q_4 & \text{if } i = 2 \text{ and } x = 0\\
      q_0 & \text{if } i = 2 \text{ and } x = 1\\
      q_1 & \text{if } i = 3 \text{ and } x = 0\\
      q_2 & \text{if } i = 3 \text{ and } x = 1\\
      q_3 & \text{if } i = 4 \text{ and } x = 0\\
      q_4 & \text{if } i = 4 \text{ and } x = 1
    \end{cases}\\
    q_0 &= q_0\\
    F &= \set{q_0}
  \end{align*}

\end{Answer}
