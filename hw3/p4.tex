\begin{problem}
  \begin{enumalph}
    \item
    For a language $A$ over alphabet $\Sigma$, define
    the relation $\equiv_A$ on strings in $\Sigma^*$ as follows:
    ``$x \equiv_A y$'' means:
    \[ \forall w \in \Sigma^* (xw \in A \Iff yw \in A). \]
    \step
    Formally (and concisely) prove that $\equiv_A$ is an equivalence relation.
    \begin{Answer}
      If $\equiv_A$ is an equivalence relation,
      then it must be reflexive, symmetric, and transitive.
      \begin{enumroman}
        \item \textbf{Reflexivity:}

        \step
          It is trivial to show that $\equiv_A$ is reflexive:
          For any string $x$, if extending $x$ with an arbitrary string $w$ makes
          it a member of $A$, then the same extension will always make $x$ a member
          of $A$.  Thus, $x \equiv_A x$.
        \item \textbf{Symmetry:}
        
        \step
          Suppose $x \equiv_A y$.
          Then, for \emph{all} strings $w \in \Sigma^*$, extending $x$ with $w$ makes it a member of $A$
          \emph{if and only if} extending $y$ with $w$ makes it a member of $A$.
          The reverse must also hold: extending $y$ with a string $w \in \Sigma^*$
          makes it a member of $A$ \emph{if and only if} extending $x$ with $w$ also makes it a member of $A$.
          Therefore, $\equiv_A x$ whenever $x \equiv_A y$.
        \item \textbf{Transitivity:}
          Let $a, b, c \in \Sigma^*$ be such that $a \equiv_A b$ and $b \equiv_A c$.
          By definition of $\equiv_A$, $aw \in A \Iff bw \in A$, and $bx \in A \Iff cx \in A$.

          \step
          Take $w$ to be an arbitrary extension of $a$ such that $aw \in A$,
          then we also have that $bw \in A$. However, since $b \equiv_A c$,
          this also means that $cw \in A$.
          Therefore, $aw \in A \Iff cw \in A$, so $a \equiv_A c$.
      \end{enumroman}
    \end{Answer}
    
    \newpage
    \item The equivalence $\equiv_A$ is called the
      \emph{left equivalence relation} of the language $A$.
      An equivalence relation on a set partitions the set into
      disjoint subsets called equivalence classes
      in the following way: two elements belong to the same
      class iff they are related by the equivalence relation.
      Thus, $\equiv_A$, which is a relation on $\Sigma^*$,
      partitions $\Sigma^*$ into equivalence classes:
      these are called the left equivalence classes of the language $A$.

      \step
      For example, consider the language
      $C = \set{x \in {0,1}^* : \abs{x} \text{is even}}$ 
      over the alphabet $\set{0,1}$.
      Convince yourself that any two even-length strings
      are related by $\equiv_C$, as are any two odd-length strings.
      Also, no odd-length string is related by $\equiv_C$ 
      to an even-length string.
      Thus, $C$ has exactly two left equivalence classes:
      (1) odd-length strings, i.e., $\set{0, 1}^* - C$,
      and (2) even-length strings, i.e., $C$.

      \step
      Similarly, convince yourself that the language $B = (01)^*$
      over the alphabet $\set{0, 1}$ has three equivalence classes,
      which are:
      (1) $B$,
      (2) $(01)^*0$, and
      (3) $\set{0, 1}^* - (B \cup (01)^*0)$.

      \step
      Describe the left equivalence classes of each of the following
      languages (no proofs required):
      \begin{enumroman}
        \item $L_1 = \set{a, aa, aaa, b, ba, baa}$ over the alphabet $\set{a, b}$.
          \begin{Answer}
            $L_1$ has four equivalence classes:\\
            (1) $\set{a, b}$,\\
            (2) $\set{aa, ba}$,\\
            (3) $\set{aaa, baa}$, and\\
            (4) $\set{a, b}^* - L_1$ 
          \end{Answer}
        \item $L_2 = a^*b^*c^*$ over the alphabet $\set{a, b, c}$.
          \begin{Answer}
            $L_2$ has four equivalence classes:\\
            (1) $a^*$,\\
            (2) $a^*b^+$,\\
            (3) $a^*b^*c^+$, and\\
            (4) $\set{a, b}^* - L_2$ 
          \end{Answer}
        \item $L_3 = (ab \cup ba)^*$ over the alphabet $\set{a, b}$.
          \begin{Answer}
            $L_3$ has three equivalence classes:\\
            (1) $(ab \cup ba)^* = L_3$,\\
            (2) $(ab \cup ba)^*a$,\\
            (3) $(ab \cup ba)^*b$,\\
            (4) $\set{a, b}^* - ((ab \cup ba)^* \cup ((ab \cup ba)^*(a \cup b)))$
          \end{Answer}
        \newpage
        \item $L_4 = \set{0^n1^n : n \geq 0}$ over the alphabet $\set{0, 1}$~\label{problem:4b4}.
          \begin{Answer}
            $L_4$ has infinite equivalence classes:
            \begin{align*}
              &(1) 0^n \text{ for any $n \geq 0$} &\text{(each $n$ generates an equivalence class)},\\
              &(2) 0^n 1^k \text{ for any $n > 0$ and $0 < k < n$} &\text{(each $(n - k)$ generates an equivalence class)}\\
              &(3) L_4 &\text{(A single equivalence class)}\\
              &(4) \set{0, 1}^* - (L_4 \cup 0^n \cup 0^n 1^k) &\text{(A single equivalence class)}
            \end{align*}
          \end{Answer}
      \end{enumroman}
  \end{enumalph}
\end{problem}
