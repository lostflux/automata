\begin{problem}
  \begin{enumalph}
    \item Draw a $3$-state DFA for the language
    \[
      D_3 = \set{ x \in \set{0, 1}^* :
      \beta(x) \text{ is divisible by } 3},
    \]
    where $\beta(x)$ is the string $x$ interpreted as a
    binary number.
    \begin{Answer}
      \begin{figure}[H]
        \centering
        \begin{tikzpicture}
          \node[state, initial, accepting] (q0) at (0, 0) {$q_1$};
          \node[state, right of=q0] (q1) {$q_2$};
          \node[state, right of=q1] (q2) {$q_3$};
          \draw (q0) edge[loop above, above] node {$0$} (q0)
                (q0) edge[bend left, above] node {$1$} (q1)
                (q1) edge[bend left, above] node {$0$} (q2)
                (q1) edge[bend left, below] node {$1$} (q0)
                (q2) edge[bend left, below] node {$0$} (q1)
                (q2) edge[loop right, right] node {$1$} (q2);
        \end{tikzpicture}
        \caption{DFA for $D_3$}
      \end{figure}
    \end{Answer}
    \item Convert the DFA into a regular expression
      using the $R_{ij}^k$ method.

      \step
      To do this systematically,
      make a big table with $9$ rows indexed by the pairs $(i, j)$
      and $4$ columns indexed by the possible values of $k \in \set{0, 1, 2, 3}$.
      Fill each cell of the table with a regular expression that generates
      the corresponding $R_{ij}^k$.
      Sometimes, you’ll obtain a complicated regular expression if you directly
      apply the equations from class. In such cases, show what the equations give you
      and only then simplify. You may use the shorthand $X^+$ to denote $XX^\ast$,
      where $X$ is an arbitrary regular expression.
      \begin{Answer}
        \begin{table}[H]
          \centering
          \resizebox{\columnwidth}{!}{%
          \begin{tabular}{|l|l|l|l|l|l|}
            \bottomrule
            $i$ & $j$ & $k = 0$ & $k = 1$ & $k = 2$ & $k = 3$ \\
            \midrule
            $q_1$ & $q_1$ & $0$   & $0^*$                           & $0^* \cup 0^*1(10^*1)^*10^*$                      & $0^* \cup (0^*1(10^*1)^*10^*) \cup (0^*1(10^*1)^*0(1 \cup 0(10^*1)^*01^*)^*0(10^*1)^*10^*)$  \\
            $q_1$ & $q_2$ & $1$   & $1 \cup 0^*1 = \crim{0^*1}$     & $0^*1 \cup 0^*1(10^*1)^* = \crim{0^*1(10^*1)^*}$  & $0^*1(10^*1)^* \cup (0^*1(10^*1)^*0(1 \cup 0(10^*1)^*0)^*0(10^*1)^*)$\\
            $q_1$ & $q_3$ & $\no$ & $\no$                           & $0^*1(10^*1)^*0$                                  & $0^*1(10^*1)^*0 \cup 0^*1(10^*1)^*0(1 \cup 0(10^*1)^*01^*)^* = \crim{0^*1(10^*1)^*0(1 \cup 0(10^*1)^*0)^*}$ \\
            $q_2$ & $q_1$ & $1$   & $10^*$                          & $10^* \cup (10^*1)^*10^* = \crim{(10^*1)^*10^*}$  & $(10^*1)^*10^* \cup (0 (1 \cup 0(10^*1)^*0)^*)^*0(10^*1)^*10^*$ \\
            $q_2$ & $q_2$ & $\no$ & $10^*1$                         & $(10^*1)^*$                                       & $(10^*1)^* \cup (0 (1 \cup 0(10^*1)^*0)^*0)$ \\
            $q_2$ & $q_3$ & $0$   & $0$                             & $0$                                               & $0 \cup (0(1 \cup 0(10^*1)^*0)^*) = \crim{0(1 \cup 0(10^*1)^*0)^*}$ \\
            $q_3$ & $q_1$ & $\no$ & $\no$                           & $0(10^*1)^*10^*$                                  & $0(10^*1)^*10^* \cup ((1 \cup 0(10^*1)^*0)^*0(10^*1)^*10^*) = \crim{(1 \cup 0(10^*1)^*0)^*0(10^*1)^*10^*}$ \\
            $q_3$ & $q_2$ & $0$   & $0$                             & $0(10^*1)^*$                                      & $0(10^*1)^* \cup ((1 \cup 0(10^*1)^*0)^*0(10^*1)^*) = \crim{(1 \cup 0(10^*1)^*0)^*0(10^*1)^*}$ \\
            $q_3$ & $q_3$ & $1$   & $1$                             & $1 \cup 0(10^*1)^*0$                              & $(1 \cup 0(10^*1)^*0)^*$ \\
            \toprule
          \end{tabular}
          }
          \caption{Regular Expressions Generated via the $R_{ij}^k$ Method}
        \end{table}

        \step
        The regular expression for the DFA is the union
        \[ \bigcup_{f \in F} R^n_{1n}. \]
        Since $F = \set{1}$, the regular expression for the DFA is
        \[ R_{11}^3 = 0^* \cup (0^*1(10^*1)^*10^*) \cup (0^*1(10^*1)^*0(1 \cup 0(10^*1)^*01^*)^*0(10^*1)^*10^*). \]
      \end{Answer}
  \end{enumalph}
\end{problem}
