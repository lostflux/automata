\begin{problem}
  \begin{enumalph}
    \item Draw a $3$-state DFA for the language
    \[
      D_3 = \set{ x \in \set{0, 1}^* :
      \beta(x) \text{ is divisible by } 3},
    \]
    where $\beta(x)$ is the string $x$ interpreted as a
    binary number.
    \begin{Answer}
      \begin{figure}[H]
        \centering
        \begin{tikzpicture}
          \node[state, initial, accepting] (q0) at (0, 0) {$q_0$};
          \node[state, right of=q0] (q1) {$q_1$};
          \node[state, right of=q1] (q2) {$q_2$};
          \draw (q0) edge[loop above, above] node {$0$} (q0)
                (q0) edge[bend left, above] node {$1$} (q1)
                (q1) edge[bend left, above] node {$0$} (q2)
                (q1) edge[bend left, below] node {$1$} (q0)
                (q2) edge[bend left, below] node {$0$} (q1)
                (q2) edge[loop right, right] node {$1$} (q2);
        \end{tikzpicture}
        \caption{DFA for $D_3$}
      \end{figure}
    \end{Answer}
    \item Convert the DFA into a regular expression
      using the $R_{ij}^k$ method.

      \step
      To do this systematically,
      make a big table with $9$ rows indexed by the pairs $(i, j)$
      and $4$ columns indexed by the possible values of $k \in \set{0, 1, 2, 3}$.
      Fill each cell of the table with a regular expression that generates
      the corresponding $R_{ij}^k$.
      Sometimes, you’ll obtain a complicated regular expression if you directly
      apply the equations from class. In such cases, show what the equations give you
      and only then simplify. You may use the shorthand $X+$ to denote $XX^\ast$,
      where $X$ is an arbitrary regular expression.
      \begin{Answer}  

      \end{Answer}
  \end{enumalph}
\end{problem}
