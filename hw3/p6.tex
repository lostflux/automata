\begin{problem}
  The connection between left equivalence and regularity is even deeper.
  \begin{enumalph}
    \item Let $A$ be a regular language over alphabet $\Sigma$.
      Prove (formally and rigorously) that $A$ has finitely many
      distinct left equivalence classes.
      The function $\delta^*$ we defined in class might be useful.
      \begin{Answer}
        Let $A$ be a regular language over an alphabet $\Sigma$.
        \begin{claim}
          $A$ has finitely many distinct left equivalence classes.

          \begin{proof}
            Since $A$ is regular, there exists a DFA
            \[ M = (Q, \Sigma, \delta, q_0, F), \]
            where $Q$ is a finite set, that recognizes $A$.

            \step
            Define the function $\delta^*: Q \times \Sigma^* \to Q$ as follows:
            \[ \delta^*(q, x) =
              \begin{cases}
                q & \text{if } x = \eps \\
                \delta^*(\delta(q, x_1), x_2 \cdots x_n) & \text{if $x = x_1 x_2 \cdots x_n$ with each $x_i \in \Sigma$}
              \end{cases}
            \]

            \step
            First, we show that each distinct equivalence class can be fully captured by a single state.
            Take any equivalence class $[x]_A \in A$.
            Then if any string $w \in A$ extends a member of $[x]_A$ to a member of $A$,
            then $w$ extends \emph{all members} of $[x]_A$ to a member of $A$.
            Therefore, we can fully capture $[x]_A$ by a single state $q$ in $M$,
            since all members of $[x]_A$ have matching sequences of transitions out of $q$
            to accepting / rejecting states.

            \step
            Next, we show that no two distinct equivalence classes can be captured by the same state.
            Take $[x]_A$ and $[y]_A$ as two distinct equivalence classes in $A$,
            such that $x_w \in A$ and $y_w \notin A$ for some $w \in \Sigma^*$.
            Suppose the two classes, $[x]_A$ and $[y]_A$, are captured by the same state $q$ in $M$.
            Since $M$ is a DFA, it always makes the same sequence of transitions out of $q$
            so it would be impossible for it to accept the string $xw$ and reject the string $yw$.
            Therefore, $[x]_A$ and $[y]_A$ cannot be captured by the same state $q$ in $M$.

            \step
            Finally, we tie up the above conditions to prove that $A$ has a finite number of
            distinct equivalence classes. Recall that $A$ is regular, and $M$ recognizes $M$.
            As shown above, each state $q \in Q$ may fully capture a distinct equivalence class,
            but it may not capture \emph{more than one} distinct equivalence class.
            This means that:
            
            \begin{corollary}~\label{cor:equiv-states}
              If $L$ is a regular language and $M_L = (Q_L, \Sigma_L, \delta_L, q_0, F_L)$
              is a DFA that recognizes $L$, then number of states in $M_L$
              is greater than or equal to the number of equivalence classes in $L$, i.e,
              if $\set{\#[x]_L : x \in L}$ is the set of \textbf{distinct} equivalence classes in $L$,
              \[ \abs{Q_L} \ge \abs{\set{\#[x]_L : x \in L}} \]
            \end{corollary}
            
            \step
            By the corollary, the number of equivalence classes of a language $A$ may not be more than
            the number of states in a DFA that recognizes $A$.
            However, $M$, our original DFA recognizing  has a finite number of states, so $A$ must have a finite number of
            distinct left equivalence classes.
          \end{proof}
        \end{claim}
      \end{Answer}
    
    \item Using one or more of the results proved above, give alternate
      proofs that the following languages are not regular:
      \begin{enumroman}
        \item $L_4 = \set{0^n1^n : n \ge 0}$.
          \begin{Answer}
            \step
            Suppose $L_4$ is regular, and $M_4$ is a DFA that recognizes $L_4$,
            By corollary~\ref{cor:equiv-states}, $M_4$ must have
            at least as many states as $L_4$ has equivalence classes.
            But, as shown in problem 4.2, part~\ref{problem:4b4}
            $L_4$ has infinite equivalence classes,
            meaning $M_4$ must have infinite states,
            and DFAs must have a \emph{finite} number of states.
            
            \step
            Therefore, no DFA can recognize $L_4$,
            so $L_4$ is not regular.
          \end{Answer}
        \item $L = \set{x \in \set{0, 1}^* : x = x^R}$.
          \begin{Answer}
            \step
            We start by noting that $L$ has infinite equivalence classes.
            After reading a string $x$, we only accept if:
            \begin{enumroman}
              \item $x \in L$
              \item we read a string $w = x^R$
              \item We read some string $w$ such that $xw = yyR$ for some $y \in \set{0, 1}^*$.
            \end{enumroman}

            \step
            Since $\set{0, 1}^*$ is infinite, there are infinite such strings $x$,
            therefore infinite equivalence classes.

            \step
            Now, suppose $L$ is regular, and $M$ is a DFA that recognizes $L$.
            By corollary~\ref{cor:equiv-states}, $M$ must have
            at least as many states as $L$ has equivalence classes.
            But, as shown above, $L$ has infinite equivalence classes,
            meaning $M$ must have infinite states,
            and DFAs must have a \emph{finite} number of states.

            \step
            Therefore, no DFA can recognize $L$,
            so $L$ is not regular.
          \end{Answer}
      \end{enumroman}
  \end{enumalph}
\end{problem}
