\begin{problem}
  Sloppy Q. Thinker and Fawlty J. Logician are two computer science
  amateurs who have not taken a rigorous course in the Theory of Computation,
  such as Dartmouth’s CS 39.
  They have come up with the following fallacious ``proofs.''

  Criticize each of these fallacious ``proofs'' thoroughly.
  How would you convince Sloppy and Fawlty that not only are they wrong,
  but ideas along these lines simply cannot work for proving what they are claiming?

  \begin{enumalph}
    \item Sloppy has this ``proof'' that $P \neq NP$.
      Consider an algorithm for \SAT:
      ``On input $\phi$, try all possible assignments to the variables.
      Accept if any satisfy $\phi$.''
      This algorithm clearly requires exponential time.
      Thus, \SAT has exponential time complexity.
      Therefore \SAT is not in $P$.
      Because \SAT is in $NP$, it must be the case that $P \neq NP$.
      \begin{Answer}
        Sloppy's solution overlooks the fact that $P \subseteq NP$.
        Many (perhaps all?) problems that can be solved in polynomial time
        by some clever algorithm can also be solved in exponential time by
        a brute-force algorithm.
        For instance, finding the minimum value in a \emph{sorted} array can be
        solved efficiently in $O(1)$ time by using the fact that the minimum value
        is the first element, yet the same problem can also be solved very inefficiently
        in $O(n^n)$ time by non-deterministically guessing a possible minimum value's index
        and comparing the value at that index to every other value in the array.
        Sloppy needs to take the necessary step of showing that \SAT cannot, in fact,
        be solved in polynomial time, such as by reducing a problem that has been
        proven to be NP-complete to \SAT hence showing that \SAT must not be
        in $P$ (hence, $P \neq NP$).

      \end{Answer}
    \item Fawlty instead has a ``proof'' that $P = NP$.
      Take an arbitrary language $L \in NP$.
      Let $N$ be a polynomial-time NDTM decider for $L$.
      Since NDTMs are equivalent to TMs, there is a deterministic TM, $M$,
      that also decides $L$.
      This shows that $L \in P$. Therefore $NP \subseteq P$.
      Since we clearly have $P \subseteq NP$, it follows that $P = NP$.
      \begin{Answer}
        Fawlty assumes that the equivalent deterministic TM $M$ also runs in
        polynomial-time.
        This is not necessarily the case --- in fact, the direct equivalent
        would run in exponential time. If indeed $L \in P$, Fawlty needs to directly
        prove the existence of a polynomial-time \emph{deterministic} TM that decides $L$.
      \end{Answer}
  \end{enumalph}
\end{problem}
