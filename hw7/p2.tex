\begin{problem}
  Thanks to the Cooke-Levin theorem, we know that
  \TColor can be polynomial-time reduced to \TSAT.
  However, for this problem, pretend that you did not know that theorem.
  Give a direct polynomial-time reduction from \TColor to \TSAT.
\end{problem}
\begin{Answer}
  Let $G = (V, E)$ be the graph in question.
  Let $R, G, B$ be predicates saying a given vertex is colored red, green,
  or blue, respectively, e.g. $Rx$ means ``vertex $x$ is colored red''.
  To model \TColor as a \TSAT instance, we first need to define axioms
  for the conditions that need to be satisfied in \TColor,
  ensuring that each clause has at most $3$ literals.

  \begin{enumroman}
    \item Each vertex $v \in V$ must be assigned a color:
      \[
        \Sigma_1 = \set{ (Rv \lor Gv \lor Bv) : v \in V }
      \]

      \step
      This construction of $\Sigma_1$ can be done in $O(|V|)$ time.

    \item Each vertex may not be assigned more than one color
      (equivalently, each vertex \emph{must} not be assigned
      $2$ of the $3$ possible colors):
      \begin{align*}
        \Sigma_{2.1} &= \set{ (\neg Rv \lor \neg Gv \lor \neg Gv) : v \in V } \\
        \Sigma_{2.2} &= \set{ (\neg Rv \lor \neg Bv \lor \neg Bv) : v \in V } \\
        \Sigma_{2.3} &= \set{ (\neg Gv \lor \neg Bv \lor \neg Bv) : v \in V } \\
        \Sigma_2 &= \Sigma_{2.1} \cup \Sigma_{2.2} \cup \Sigma_{2.3}
      \end{align*}

      \step
      This construction of $\Sigma_2$ can be done in $O(|V|)$ time.
    \item Adjacent vertices are not colored the same color:
      \begin{align*}
        \Sigma_{3.1} &= \set{ (\neg Rv \lor \neg Ru \lor \neg Ru) : (u, v) \in E } \\
        \Sigma_{3.2} &= \set{ (\neg Gv \lor \neg Gu \lor \neg Gu) : (u, v) \in E } \\
        \Sigma_{3.3} &= \set{ (\neg Bv \lor \neg Bu \lor \neg Bu) : (u, v) \in E } \\
        \Sigma_3 &= \Sigma_{3.1} \cup \Sigma_{3.2} \cup \Sigma_{3.3}
      \end{align*}

      \step
      This construction of $\Sigma_3$ can be done in $O(|E|)$ time.
  \end{enumroman}

  \step
  Finally, let $\Sigma = \Sigma_1 \cup \Sigma_2 \cup \Sigma_3$,
  and define the formula
  \[
    \psi = \bigwedge_{\sigma \in \Sigma} \sigma
  \]
  This construction of $\phi$ can be done in $O(|V| + |E|)$ time.

  \step
  We can thus define a reduction from \TColor to \TSAT in
    $O(\abs{V} + \abs{E})$ time as follows:

  \step
  ``On input $G = (V, E)$, where $G$ is a graph:
  \begin{enumarabic}
    \item Construct the formula $\psi$ from $G$ as described above.
    \item Output $\psi$.''
  \end{enumarabic}

  \step
  \begin{align*}
    G \in \TColor &\Iff \exists \text{ a valid $3$-coloring of $G$ } \\
    &\Iff \exists \text{ a coloring of $G$ satisfying $\psi$ } \\
    &\Iff \psi \text{ is satisfiable } \\
    &\Iff \psi \in \TSAT
  \end{align*}


\end{Answer}
