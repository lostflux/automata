\begin{problem}
  Recall that the theory of time complexity was developed using
  worst case cost: for a particular length $n$, we used the maximum
  of all time costs for inputs of length up to $n$.
  Why not the average?
  This exercise will show you one good reason why not:
  the natural notion of ``average'' fails to properly capture
  the hardness of a computational problem.
  
  Let $M$ be a decider TM with input alphabet $\Sigma$.
  Recall that we defined $\TCost(x)$, for $x \in \Sigma^*$,
  to be the number of steps $M$ takes on input $x$ until it halts.
  Let us now define the function $\AvgTCost : \N \to \N$ as follows:
  \[
    \AvgTCost(n) = \frac{1}{\abs{\Sigma}^n} \sum_{x \in \Sigma \,:\, \abs{x} = n} \TCost(x).
  \]
  The corresponding class of algorithms that run in
  ``polynomial time on average'' would be
  \[
    \AvgP = \set{ L \subseteq \Sigma^* :
    \text{$\exists$ decider $M$ and integer $k$
    such that $\cL(M) = L$ and $\AvgTCost(n) = O(n^k)$} }.
  \]
  Construct a language $L$ such that on the one hand $L$ is $NP$-complete,
  (strongly suggesting that it is hard) and, on the other hand,
  $L \in \AvgP$ (misleadingly suggesting that it is easy).
  
  \step
  \grey{
    Hint: Suitably modify the langauge \SAT.
  }
\end{problem}

\begin{Answer}
  Consider the language
  \[ \WSAT = \set{ x 0^{\abs{x}} : x \in \SAT }. \]

  \bigskip
  To see that \WSAT is NP-complete, consider the following
  reduction of \SAT to \WSAT, that takes well-formed strings
  in \WSAT and returns the corresponding prefix (that is required to be in \SAT):

  \step
  $T_0$ = ``On input $\vector{w0^{\abs{w}}}$:
    \begin{enumarabic}
      \item \Return $\vector{w}$.''
    \end{enumarabic}
  
  \step
  Clearly, $w \in \SAT \Iff w0^{\abs{w}} \in \WSAT$.
  Since \SAT is NP-complete, \WSAT must also be NP-complete
  since if we could efficiently solve \WSAT we could use the algorithm
  to derive efficient solutions to \SAT.

  \bigskip
  Now, consider the average-case runtime costs of any recognizer $M$ for \WSAT
  on inputs $x$ in \WSAT.
  We know \SAT is NP-complete, let $T_{\SAT}$ be an NDTM that solves \SAT.

  \step
  $T_1$ = ``on input $\vector{w}$:
    \begin{enumarabic}
      \item Determine if $w$ is of the form $x0^{\abs{x}}$ for some $x \in \Sigma^*$.
        If not, \Reject.
      \item If yes, write $w$ as$x0^{\abs{x}}$, and run $T_{\SAT}$ on input $\vector{x}$.
      \item If $T_{\SAT}$ accepts, \Accept, otherwise \Reject.''
    \end{enumarabic}

  \step
  Now let's consider the running time of $T_1$.
  Let $n$ be the size of input strings.
  Step one determines if the string can be broken down into two equal parts,
  with the second part being all zeros.
  This can be done in $O(n)$ time as it only requires scanning the string
  a constant number of times.
  Step $2$ then tests if the prefix $x$ is in \SAT.
  This requires testing all possible assignments to the variables in $x$,
  since \SAT is NP-complete. Thus, this takes
  $2^{\abs{x}} = 2^{n/2} = O(2^n)$ time.

  \step
  Now, let's consider the average-case running time of $T_1$
  on strings of fixed lengths $n$.
  \begin{enumroman}
    \item $n$ is odd:
      Clearly, a string with odd length cannot be written as two strings of equal length.
      Therefore, $T_1$ only needs to run the checker for step $1$, which
      takes $O(n)$ time.
    \item $n$ is even:
      If a string of length $n$ is in \WSAT
      then its second half must be of the form $0^{n/2}$.
      Given an alphabet of size $m$, there are a total of $m^{n}$ strings of length $n$,
      and only $m^{n/2}$ strings are plausible members of \WSAT
      (contingent on checking that the prefix is in \SAT).
      Thus, the running time will be $O(2^{n/2}) = O(2^n)$ for \emph{each} of the $m^{n/2}$ strings,
      and $O(n)$ for the remaining $m^n - m^{n/2}$ strings.
      Thus;
      \begin{align*}
        \AvgTCost(n) &= \frac{1}{\abs{\Sigma}^n} \sum_{x \in \Sigma \,:\, \abs{x} = n} \TCost(x) \\
        &= \frac{(m^n - m^{n/2}) n + m^{n/2} 2^{n/2}}{m^n}  \\
        &= \frac{(m^n - m^{n/2})}{m^n} n + \frac{m^{n/2}}{m^n} 2^{n/2} \\ 
        &= n - \frac{m^{n/2}}{m^n} n + \frac{m^{n/2}}{m^n} 2^{n/2} \\
        &= O(n) + \parens{\frac{2}{m}}^n \\
        m \ge 2 &\Rightarrow \AvgTCost(n) = O(n)
      \end{align*}
  \end{enumroman}
\end{Answer}
