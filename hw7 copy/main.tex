\documentclass[11pt, reqno]{amsart}

% Include the macros file from `../common'
\input{~/latex-common/macros.tex}

\renewcommand{\theenumi}{\alph{enumi}}

\begin{document}
\setlength{\headheight}{13.0pt}
\setlength{\footskip}{13.0pt}




% TITLE
\newdate{due-date}{06}{02}{2023}
\statement{}{Winter `23}{Chakrabarti}{Amittai Siavava}{CS 39: Theory of Computation}

%CREDIT STATEMENT
% \CreditStatement{
%   I discussed ideas for this homework assignment with Paul Shin.
%   I also Boxian's office hours wherein we discussed
%   some general approaches to some of the problems.
%   \step
%   I  referred to the following books:
%   \begin{enumerate}
%     \item \textbf{Introduction to the Theory of Computation} by \textbf {Michael Sipser}.
%     % \item \textbf{A Mathematical Introduction to Logic} by \textbf {Herbert Enderton}.
%   \end{enumerate}
% }



\pagestyle{fancy}                       % fancy (allow headers, footers)
\fancyhf{}                              % clear all header/footer settings.
\cfoot{\thepage}                        % set page-numbers in footer.
\lhead{\textit{\textbf{ Amittai, S}}}   % set name in header, left.
\rhead{\textsc{CS 39: Theory of Computation}}         % set class name in header, right.

\bigskip

My primary interests in computer science are in systems engineering
and in deep learning, and I had seen finite automata show up in both
areas.

\begin{enumarabic}
  \item CS-51 (Computer Architecture), one of the projects
    entailed constructing a CPU using circuits, and to make it work we had
    to program a finite state machine that performs all the micro-operations
    For example, a simple instruction such as adding the contents
    of two memory addresses and storing the result into a third memory
    address would be broken down into a sequence of transitions,
    each of which does a single thing (move A, move B, add A and B,
    store ALU result in memory, loop until the memory write is finished...)
    This was very similar to the random access machine that
    we later discussed in CS-39, and this class gave me a better understanding
    of how such computation works.

 \item One of my other areas of interest is deep learning, and
    One of the current issues is that although we can
    change parameters, refine the training data, etc., to improve the
    performance of deep neural networks, most of the internal workings
    is not fully understood --- akin to a black-box that you tune then
    feed in input and get output, which, usually, happens to have high accuracy.
    Late last year, a team of German researchers experimented with modeling
    deep neural networks with finite automata that the input-output behavior
    of the neural network. Studying these finite automata and how
    they change when the neural network's training parameters are modified
    could then give insights into neural network itself.
\end{enumarabic}

That aside, decidability and computability were the most intriguing
topics to me. They also both showed up in one of my other classes,
Math 69 (Mathematical Logic) and it was amazing to see two disparate
vantage points converge to the same ideas.
One of the ways I engaged with the course content was through
weekly discussions I had with Paul Shin, one of my classmates.
We would  meet on weekends to talk about the problem sets and often spent
hours talking about the course content in general and related problems.
For example, he was doing a directed study in category theory this term,
and I have some experience with functional programming which uses ideas from
category theory to model computation.
These discussions particularly resonated with me because I do not always
feel comfortable speaking up in class where there's multiple other students.

I did not get time to implement the machines we were studying in class
due to the volume of work I had this term. It is something I am excited
to do over the break.
\vfill

\end{document}
