\begin{problem}
  Define the \emph{reversal} of a language $L$ to be the language
  $L^R := \set{x^R : x \in L}$.
  \begin{enumalph}
    \item Let $M = \parens(Q, \Sigma, \delta, q_0, F)$ be a DFA.
      Give a formal description of an NFA that recognizes $\cL(M)^R$.
      Give brief proofs that your construction is complete and sound.
    \begin{Answer}
      $N = (Q \cup \set{ s}, \Sigma, \delta_2, s, F_2)$ where $s \notin Q$, and:
      \begin{align*}
        \delta_2(s, \eps) = F \\
        \delta_2(q, a) = \set{q' : \delta(q', a) = q'} \\
        F_2 = \set{ q_0 }
      \end{align*}

      \step
      Completeness:

      Let $w \in \cL(M)$ be a string recognized by $M$. Let $n = \abs{x}$, then there
      exists a computational path $x_1, x_2, \ldots x_n$ of $M$ on $w$ such that
      $x_1 = q_0$, $\delta(x_j, w_j) = x_{j+1}$ for all $j < n$, and $x_n \in F$.
      By the construction of $N$ from $M$, there exists an $\eps$-transition from
      $s$ to $x_n$ and there exists a computation path of $N$ on $w$ that takes
      $N$ from $x_n$ to $x_1$. Consequently, there exists a computation path of $N$
      on $w^R$ that takes $N$ from $s$ to $q_0$, which is an accepting state of $N$,
      so $N$ accepts $w^R$.

      \step
      Soundness:

      Suppose $w = w_1 w_2 \ldots w_n$ is a string accepted by $N$.
      Then there exists some computation path $x_1, x_2, \ldots x_k$ of $N$ on $w$
      such that $x_1 = s$, $x_2 \in F$ (where $F$ is the set of accepting states
      in the original DFA), $\delta_2(x_j, w_j) = x_{j+1}$ for all $j < k$, and
      $x_k \in F_2$. Ignoring the epsilon transition from $s$ to $x_2$,
      we have that $\delta(x_{j+1}, w_j) = x_j$ for all $j < k$,
      where $\delta$ is the original transition function of $M$.
      Since $x_k = q_0$, this means there is some computation path of $M$ on $w$
      that starts at $q_0$ and ends at $x_2$, which happens to be an accepting state
      in $M$. Therefore, $M$ accepts $w^R$.
    \end{Answer}

    \newpage
    \item Let $k \geq 3$ be an integer. Prove that there exists a $k$-state
      DFA $M_k$ over the alphabet $\set{0, 1}$ such that every DFA that
      recognizes $\cL(M_k)^R$ requires at least $2^{k-2}$ states.

      \step
      \grey{
        Hint: Look to the languages we considered very early in the course
        for inspiration: you’re looking for a language that admits a small DFA
        whose reversal requires an exponentially larger DFA.
        For the proof of the $2^{k-2}$ lower bound, use the techniques and results
        you learnt in Homework 3.
        If you can’t quite prove the lower bound as stated, you can still earn
        almost full credit for proving a $2^{\Omega(k)}$ bound.
      }
      \begin{Answer}
        Let $L = (0 \cup 1)^{k-2}1(0 \cup 1)^*$.
        Then $L$ is the language of strings in which the $k-2$th character is a $1$.
        This is easily recognizable by a DFA with only $k$ states:
        \begin{align*}
          M = (Q, \set{0, 1}, \delta, q_0, F) \text{ where}\\
          Q = \set{q_i : 0 \leq i < k-2} \cup \set{ q_{accept}, q_{reject}},\\
          F = \set{q_{accept}},\\
          \delta(q, x) = \begin{cases}
            q_{accept} & \text{if } x = 1 \text{ and } q = q_{k-3} \\
            q_{accept} & \text{if } q = q_{accept} \text{ and } x \in \set{0, 1} \\ \\
            q_{reject} & \text{if } x = 0 \text{ and } q = q_{k-3} \\
            q_{reject} & \text{if } q = q_{reject} \\ \\
            q_{i+1} & \text{if } x \in \set{0, 1} \text{ and } q = q_i \text{ and } i < k-2 \\
            q_{reject} & \text{otherwise (in case of invalid strings!)}
          \end{cases}
        \end{align*}

        The idea is to use states to count the number of characters read in from the input.
        We make $k-2$ transitions from state $q_0$ to state $q_{k-3}$.
        We then check whether the letter $k-1$ is a $0$ or a $1$ and accept or reject accordingly.
        Both $q_{accept}$ and $q_{reject}$ are trap states; the remainder of the string
        is only checked to make sure it is a valid string in the language.

        \newpage
        Now, consider the reversal of $M$.
        That is, the DFA recognizing the language \[ \cL(M)^R = (0 \cup 1)^*1(0 \cup 1)^{k-2}. \]
        Let $M^R$ be a DFA recognizing $\cL(M)^R$.
        Since the length of the input string is not known ahead of time,
        $M^R$ must track the previous $k-2$ characters read in from the input
        at any given point. For instance, if $k-1 = 4$ (i.e. we want the $4$-th last character
        to be a $1$), then each of $\set{000, 001, 010, 011, 100, 101, 110, 111}$ is in a
        unique equivalence class and must be tracked separately since reading some sequence
        in each case, reading some sequence of characters results in a unique sequence
        of transitions.

        In total, the last $k-2$ characters characters must be tracked in order to determine
        the letter at position $k-1$ from the end after any sequence of transitions,
        and there are a 2 possibilities for each of the $k-2$ positions in the string that
        are being tracked, so $M^R$ must have at least $2^{k-2}$ states.

      
      \end{Answer}
  \end{enumalph}
\end{problem}
