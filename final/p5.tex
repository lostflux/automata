
\def \Edfa { \textsc{E}_{\textsc{DFA}}}
\begin{problem}
  We showed in class that the language $\EqDFA$ is decidable.
  In this problem, for concreteness’ sake, assume that $\EqDFA$ only concerns
  DFAs over the alphabet $\set{0, 1}$ and that DFAs are encoded in such a way
  that $\EqDFA \subseteq \set{0, 1}^*$. \\
  Alice asserts that $\EqDFA \in \sfP$, while Bob asserts that
  $\EqDFA$ is $\sfNP$-complete.
  Which of them is right? Prove your answer.
\end{problem}
\begin{Answer}
  Alice is right that $\EqDFA \in \sfP$.

  Note that
  \[ \Edfa := \set{\vector{M} : \text{$M$ is a DFA and $\cL(M) = \emptyset$}} \]
  can be decided as follows:

  \step
  $T_1$ = ``On input $\vector{M}$, where $M = (Q, \Sigma, \delta, q_0, F)$ is a DFA:
  \begin{enumarabic}
    \item Simulate a graph search from $q_0$,
      taking every possible transition and ignoring loops until every possible
      path dies.
      \item If any state $q_f \in F$ is reachable, then there exists some
      accepting computational path for some string in the DFA's language,
      so $\vector{M} \not \in \Edfa$. \Reject.
      \item If no state $q_f \in F$ is reachable, then is no accepting path
      of $M$ on any string in its alphabet, hence $\cL(M) = \emptyset$.
      \Accept.''
  \end{enumarabic}
  Note that since $T_1$ does not revisit states it runs in $O(n + m)$
  where $n = \abs{Q}$ and $m$ is the number of transitions in $M$.

  \step
  We can decide $\EqDFA$ as follows:

  \step
  $T_2$ = ``On input $\vector{M_1, M_2}$, where $M_1$ and $M_2$ are DFAs:
    \begin{enumalph}
      \item Construct a new DFA $M$ such that 
      $\cL(M) = (\cL(M_1) \setminus \cL(M_2)) \cup (\cL(M_2) \setminus \cL(M_1))$.
      \item Simulate the decider for $\Edfa$, $T_1$, on $\vector{M}$.
      \item If $T_1$ accepts $\vector{M}$, then $\cL(M_1) = \cL(M_2)$. \Accept.
      \item If $T_1$ rejects $\vector{M}$, then $\cL(M_1) \neq \cL(M_2)$. \Reject.''
    \end{enumalph}

    \newpage
  \begin{claim}
    $T_2$ runs in polynomial time.

    \begin{proof}
      First, we show that the construction of $M$ runs in polynomial time.
      Given $M_1 = (Q_1, \Sigma_1, \delta_1, q_{01}, F_1)$ and
      $M_2 = (Q_2, \Sigma_2, \delta_2, q_{02}, F_2)$, we construct $M$ as follows:
      \begin{enumroman}
        \item Define $M = \set{ (q_i, q_j) : q_i \in Q_1, q_j \in Q_2}$.
        \item Define $q_0 = (q_{01}, q_{02})$.
        \item Define $\delta((q_i, q_j), a) = (\delta_1(q_i, a), \delta_2(q_j, a))$.
        \item Define $F = \set{(q_i, q_j) : q_i \in F_1 \text{ and } q_j \not \in F_2
          \text{ or } q_i \not \in F_1 \text{ and } q_j \in F_2}$.
      \end{enumroman}

      \step
      If $M_1$ has $n_1$ states and $m_1$ transitions, and $M_2$ has $n_2$ stats
      and $m_2$ transitions, then $M$ has $n_1 \cdot n_2$ states
      and $m_1 \cdot m_2$ transitions. Thus, this operation runs in polynomial time
      (since its runtime is just the sum of all the states and transitions being created).

      $T_2$ then simulates $T_1$ on $M$. Since $T_1$ just performs a breadth-first
      graph search, it runs in linear time in the sum of the number of states
      and the number of transitions in $M$, which we showed above to be polynomial.
      Since, $T_1$ runs in polynomial time, $\EqDFA \in \sfP$.
    \end{proof}
  \end{claim}
\end{Answer}


